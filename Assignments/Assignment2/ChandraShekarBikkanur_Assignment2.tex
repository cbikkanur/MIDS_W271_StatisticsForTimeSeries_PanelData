\documentclass[]{article}
\usepackage{lmodern}
\usepackage{amssymb,amsmath}
\usepackage{ifxetex,ifluatex}
\usepackage{fixltx2e} % provides \textsubscript
\ifnum 0\ifxetex 1\fi\ifluatex 1\fi=0 % if pdftex
  \usepackage[T1]{fontenc}
  \usepackage[utf8]{inputenc}
\else % if luatex or xelatex
  \ifxetex
    \usepackage{mathspec}
  \else
    \usepackage{fontspec}
  \fi
  \defaultfontfeatures{Ligatures=TeX,Scale=MatchLowercase}
\fi
% use upquote if available, for straight quotes in verbatim environments
\IfFileExists{upquote.sty}{\usepackage{upquote}}{}
% use microtype if available
\IfFileExists{microtype.sty}{%
\usepackage{microtype}
\UseMicrotypeSet[protrusion]{basicmath} % disable protrusion for tt fonts
}{}
\usepackage[margin=1in]{geometry}
\usepackage{hyperref}
\hypersetup{unicode=true,
            pdftitle={W271 - Assignment2},
            pdfauthor={Chandra Shekar Bikkanur},
            pdfborder={0 0 0},
            breaklinks=true}
\urlstyle{same}  % don't use monospace font for urls
\usepackage{color}
\usepackage{fancyvrb}
\newcommand{\VerbBar}{|}
\newcommand{\VERB}{\Verb[commandchars=\\\{\}]}
\DefineVerbatimEnvironment{Highlighting}{Verbatim}{commandchars=\\\{\}}
% Add ',fontsize=\small' for more characters per line
\usepackage{framed}
\definecolor{shadecolor}{RGB}{248,248,248}
\newenvironment{Shaded}{\begin{snugshade}}{\end{snugshade}}
\newcommand{\AlertTok}[1]{\textcolor[rgb]{0.94,0.16,0.16}{#1}}
\newcommand{\AnnotationTok}[1]{\textcolor[rgb]{0.56,0.35,0.01}{\textbf{\textit{#1}}}}
\newcommand{\AttributeTok}[1]{\textcolor[rgb]{0.77,0.63,0.00}{#1}}
\newcommand{\BaseNTok}[1]{\textcolor[rgb]{0.00,0.00,0.81}{#1}}
\newcommand{\BuiltInTok}[1]{#1}
\newcommand{\CharTok}[1]{\textcolor[rgb]{0.31,0.60,0.02}{#1}}
\newcommand{\CommentTok}[1]{\textcolor[rgb]{0.56,0.35,0.01}{\textit{#1}}}
\newcommand{\CommentVarTok}[1]{\textcolor[rgb]{0.56,0.35,0.01}{\textbf{\textit{#1}}}}
\newcommand{\ConstantTok}[1]{\textcolor[rgb]{0.00,0.00,0.00}{#1}}
\newcommand{\ControlFlowTok}[1]{\textcolor[rgb]{0.13,0.29,0.53}{\textbf{#1}}}
\newcommand{\DataTypeTok}[1]{\textcolor[rgb]{0.13,0.29,0.53}{#1}}
\newcommand{\DecValTok}[1]{\textcolor[rgb]{0.00,0.00,0.81}{#1}}
\newcommand{\DocumentationTok}[1]{\textcolor[rgb]{0.56,0.35,0.01}{\textbf{\textit{#1}}}}
\newcommand{\ErrorTok}[1]{\textcolor[rgb]{0.64,0.00,0.00}{\textbf{#1}}}
\newcommand{\ExtensionTok}[1]{#1}
\newcommand{\FloatTok}[1]{\textcolor[rgb]{0.00,0.00,0.81}{#1}}
\newcommand{\FunctionTok}[1]{\textcolor[rgb]{0.00,0.00,0.00}{#1}}
\newcommand{\ImportTok}[1]{#1}
\newcommand{\InformationTok}[1]{\textcolor[rgb]{0.56,0.35,0.01}{\textbf{\textit{#1}}}}
\newcommand{\KeywordTok}[1]{\textcolor[rgb]{0.13,0.29,0.53}{\textbf{#1}}}
\newcommand{\NormalTok}[1]{#1}
\newcommand{\OperatorTok}[1]{\textcolor[rgb]{0.81,0.36,0.00}{\textbf{#1}}}
\newcommand{\OtherTok}[1]{\textcolor[rgb]{0.56,0.35,0.01}{#1}}
\newcommand{\PreprocessorTok}[1]{\textcolor[rgb]{0.56,0.35,0.01}{\textit{#1}}}
\newcommand{\RegionMarkerTok}[1]{#1}
\newcommand{\SpecialCharTok}[1]{\textcolor[rgb]{0.00,0.00,0.00}{#1}}
\newcommand{\SpecialStringTok}[1]{\textcolor[rgb]{0.31,0.60,0.02}{#1}}
\newcommand{\StringTok}[1]{\textcolor[rgb]{0.31,0.60,0.02}{#1}}
\newcommand{\VariableTok}[1]{\textcolor[rgb]{0.00,0.00,0.00}{#1}}
\newcommand{\VerbatimStringTok}[1]{\textcolor[rgb]{0.31,0.60,0.02}{#1}}
\newcommand{\WarningTok}[1]{\textcolor[rgb]{0.56,0.35,0.01}{\textbf{\textit{#1}}}}
\usepackage{graphicx,grffile}
\makeatletter
\def\maxwidth{\ifdim\Gin@nat@width>\linewidth\linewidth\else\Gin@nat@width\fi}
\def\maxheight{\ifdim\Gin@nat@height>\textheight\textheight\else\Gin@nat@height\fi}
\makeatother
% Scale images if necessary, so that they will not overflow the page
% margins by default, and it is still possible to overwrite the defaults
% using explicit options in \includegraphics[width, height, ...]{}
\setkeys{Gin}{width=\maxwidth,height=\maxheight,keepaspectratio}
\IfFileExists{parskip.sty}{%
\usepackage{parskip}
}{% else
\setlength{\parindent}{0pt}
\setlength{\parskip}{6pt plus 2pt minus 1pt}
}
\setlength{\emergencystretch}{3em}  % prevent overfull lines
\providecommand{\tightlist}{%
  \setlength{\itemsep}{0pt}\setlength{\parskip}{0pt}}
\setcounter{secnumdepth}{0}
% Redefines (sub)paragraphs to behave more like sections
\ifx\paragraph\undefined\else
\let\oldparagraph\paragraph
\renewcommand{\paragraph}[1]{\oldparagraph{#1}\mbox{}}
\fi
\ifx\subparagraph\undefined\else
\let\oldsubparagraph\subparagraph
\renewcommand{\subparagraph}[1]{\oldsubparagraph{#1}\mbox{}}
\fi

%%% Use protect on footnotes to avoid problems with footnotes in titles
\let\rmarkdownfootnote\footnote%
\def\footnote{\protect\rmarkdownfootnote}

%%% Change title format to be more compact
\usepackage{titling}

% Create subtitle command for use in maketitle
\providecommand{\subtitle}[1]{
  \posttitle{
    \begin{center}\large#1\end{center}
    }
}

\setlength{\droptitle}{-2em}

  \title{W271 - Assignment2}
    \pretitle{\vspace{\droptitle}\centering\huge}
  \posttitle{\par}
    \author{Chandra Shekar Bikkanur}
    \preauthor{\centering\large\emph}
  \postauthor{\par}
      \predate{\centering\large\emph}
  \postdate{\par}
    \date{10/6/2019}


\begin{document}
\maketitle

\begin{Shaded}
\begin{Highlighting}[]
\CommentTok{#Loading some libraries for this assignment}
\KeywordTok{library}\NormalTok{(car)}
\KeywordTok{library}\NormalTok{(dplyr)}
\KeywordTok{library}\NormalTok{(Hmisc)}
\KeywordTok{library}\NormalTok{(ggplot2)}
\KeywordTok{library}\NormalTok{(mcprofile)}
\KeywordTok{library}\NormalTok{(nnet)}
\KeywordTok{library}\NormalTok{(MASS)}
\end{Highlighting}
\end{Shaded}

\hypertarget{strategic-placement-of-products-in-grocery-stores}{%
\section{1. Strategic Placement of Products in Grocery
Stores}\label{strategic-placement-of-products-in-grocery-stores}}

Let us load the data into a data frame and do the initial EDA of the
data

\begin{Shaded}
\begin{Highlighting}[]
\NormalTok{cereal <-}\StringTok{ }\KeywordTok{read.csv}\NormalTok{(}\StringTok{"cereal_dillons.csv"}\NormalTok{, }\DataTypeTok{header=}\OtherTok{TRUE}\NormalTok{, }\DataTypeTok{sep=}\StringTok{","}\NormalTok{)}
\KeywordTok{head}\NormalTok{(cereal, }\DecValTok{5}\NormalTok{)}
\end{Highlighting}
\end{Shaded}

\begin{verbatim}
##   ID Shelf                               Cereal size_g sugar_g fat_g
## 1  1     1 Kellog's Razzle Dazzle Rice Crispies     28      10     0
## 2  2     1            Post Toasties Corn Flakes     28       2     0
## 3  3     1                Kellogg's Corn Flakes     28       2     0
## 4  4     1               Food Club Toasted Oats     32       2     2
## 5  5     1                     Frosted Cheerios     30      13     1
##   sodium_mg
## 1       170
## 2       270
## 3       300
## 4       280
## 5       210
\end{verbatim}

\begin{Shaded}
\begin{Highlighting}[]
\KeywordTok{str}\NormalTok{(cereal)}
\end{Highlighting}
\end{Shaded}

\begin{verbatim}
## 'data.frame':    40 obs. of  7 variables:
##  $ ID       : int  1 2 3 4 5 6 7 8 9 10 ...
##  $ Shelf    : int  1 1 1 1 1 1 1 1 1 1 ...
##  $ Cereal   : Factor w/ 38 levels "Basic 4","Capn Crunch",..: 17 34 19 13 16 9 2 3 30 8 ...
##  $ size_g   : int  28 28 28 32 30 31 27 27 29 33 ...
##  $ sugar_g  : int  10 2 2 2 13 11 12 9 11 2 ...
##  $ fat_g    : num  0 0 0 2 1 0 1.5 2.5 0.5 0 ...
##  $ sodium_mg: int  170 270 300 280 210 180 200 200 220 330 ...
\end{verbatim}

We see that \emph{Shelf} is of integer type. This should be changed to a
\emph{factor} data type to do any regressions on the data.

\textbf{1.1 (1 point):} The explanatory variables need to be reformatted
before proceeding further (sample code is provided in the textbook).
First, divide each explanatory variable by its serving size to account
for the different serving sizes among the cereals. Second, rescale each
variable to be within 0 and 1. Construct side-by-side box plots with dot
plots overlaid for each of the explanatory variables. Also, construct a
parallel coordinates plot for the explanatory variables and the shelf
number. Discuss whether possible content differences exist among the
shelves.

\begin{Shaded}
\begin{Highlighting}[]
\NormalTok{stand01 <-}\StringTok{ }\ControlFlowTok{function}\NormalTok{(x) \{ (x }\OperatorTok{-}\StringTok{ }\KeywordTok{min}\NormalTok{(x))}\OperatorTok{/}\NormalTok{(}\KeywordTok{max}\NormalTok{(x) }\OperatorTok{-}\StringTok{ }\KeywordTok{min}\NormalTok{(x)) \} }\CommentTok{# function to standardize a dataset}
\NormalTok{cereal2 <-}\StringTok{ }\KeywordTok{data.frame}\NormalTok{(}\DataTypeTok{Shelf =}\NormalTok{ cereal}\OperatorTok{$}\NormalTok{Shelf, }\DataTypeTok{sugar =} \KeywordTok{stand01}\NormalTok{(}\DataTypeTok{x =}\NormalTok{ cereal}\OperatorTok{$}\NormalTok{sugar_g}\OperatorTok{/}\NormalTok{cereal}\OperatorTok{$}\NormalTok{size_g), }\DataTypeTok{fat =} \KeywordTok{stand01}\NormalTok{(}\DataTypeTok{x =}\NormalTok{ cereal}\OperatorTok{$}\NormalTok{fat_g}\OperatorTok{/}\NormalTok{cereal}\OperatorTok{$}\NormalTok{size_g), }\DataTypeTok{sodium =} \KeywordTok{stand01}\NormalTok{(}\DataTypeTok{x =}\NormalTok{ cereal}\OperatorTok{$}\NormalTok{sodium_mg}\OperatorTok{/}\NormalTok{cereal}\OperatorTok{$}\NormalTok{size_g)) }\CommentTok{# new data frame consisting of Shelf, sugar, fat and sodium}
\KeywordTok{str}\NormalTok{(cereal2)}
\end{Highlighting}
\end{Shaded}

\begin{verbatim}
## 'data.frame':    40 obs. of  4 variables:
##  $ Shelf : int  1 1 1 1 1 1 1 1 1 1 ...
##  $ sugar : num  0.643 0.129 0.129 0.112 0.78 ...
##  $ fat   : num  0 0 0 0.675 0.36 ...
##  $ sodium: num  0.567 0.9 1 0.817 0.653 ...
\end{verbatim}

\begin{Shaded}
\begin{Highlighting}[]
\KeywordTok{tail}\NormalTok{(cereal2, }\DecValTok{5}\NormalTok{)}
\end{Highlighting}
\end{Shaded}

\begin{verbatim}
##    Shelf     sugar       fat    sodium
## 36     4 0.3483871 0.3483871 0.1956989
## 37     4 0.4581818 0.5890909 0.1696970
## 38     4 0.6218182 0.3927273 0.4412121
## 39     4 0.5563636 0.5890909 0.4751515
## 40     4 0.1800000 0.0000000 0.6222222
\end{verbatim}

\begin{Shaded}
\begin{Highlighting}[]
\KeywordTok{par}\NormalTok{(}\DataTypeTok{mfrow=}\KeywordTok{c}\NormalTok{(}\DecValTok{1}\NormalTok{,}\DecValTok{3}\NormalTok{))}
\KeywordTok{boxplot}\NormalTok{(}\DataTypeTok{formula =}\NormalTok{ sugar }\OperatorTok{~}\StringTok{ }\NormalTok{Shelf, }\DataTypeTok{data =}\NormalTok{ cereal2, }\DataTypeTok{ylab =} \StringTok{"Sugar"}\NormalTok{, }\DataTypeTok{xlab =} \StringTok{"Shelf"}\NormalTok{, }\DataTypeTok{pars =} \KeywordTok{list}\NormalTok{(}\DataTypeTok{outpch=}\OtherTok{NA}\NormalTok{)) }
\KeywordTok{stripchart}\NormalTok{(}\DataTypeTok{x =}\NormalTok{ cereal2}\OperatorTok{$}\NormalTok{sugar }\OperatorTok{~}\NormalTok{cereal2}\OperatorTok{$}\NormalTok{Shelf, }\DataTypeTok{lwd =} \DecValTok{1}\NormalTok{, }\DataTypeTok{col =} \StringTok{"red"}\NormalTok{, }\DataTypeTok{method =} \StringTok{"jitter"}\NormalTok{, }\DataTypeTok{vertical =} \OtherTok{TRUE}\NormalTok{, }\DataTypeTok{pch =} \DecValTok{1}\NormalTok{, }\DataTypeTok{add =} \OtherTok{TRUE}\NormalTok{)}
\KeywordTok{boxplot}\NormalTok{(}\DataTypeTok{formula =}\NormalTok{ fat }\OperatorTok{~}\StringTok{ }\NormalTok{Shelf, }\DataTypeTok{data =}\NormalTok{ cereal2, }\DataTypeTok{ylab =} \StringTok{"fat"}\NormalTok{, }\DataTypeTok{xlab =} \StringTok{"Shelf"}\NormalTok{, }\DataTypeTok{pars =} \KeywordTok{list}\NormalTok{(}\DataTypeTok{outpch=}\OtherTok{NA}\NormalTok{)) }
\KeywordTok{stripchart}\NormalTok{(}\DataTypeTok{x =}\NormalTok{ cereal2}\OperatorTok{$}\NormalTok{fat }\OperatorTok{~}\NormalTok{cereal2}\OperatorTok{$}\NormalTok{Shelf, }\DataTypeTok{lwd =} \DecValTok{1}\NormalTok{, }\DataTypeTok{col =} \StringTok{"red"}\NormalTok{, }\DataTypeTok{method =} \StringTok{"jitter"}\NormalTok{, }\DataTypeTok{vertical =} \OtherTok{TRUE}\NormalTok{, }\DataTypeTok{pch =} \DecValTok{1}\NormalTok{, }\DataTypeTok{add =} \OtherTok{TRUE}\NormalTok{)}
\KeywordTok{boxplot}\NormalTok{(}\DataTypeTok{formula =}\NormalTok{ sodium }\OperatorTok{~}\StringTok{ }\NormalTok{Shelf, }\DataTypeTok{data =}\NormalTok{ cereal2, }\DataTypeTok{ylab =} \StringTok{"sodium"}\NormalTok{, }\DataTypeTok{xlab =} \StringTok{"Shelf"}\NormalTok{, }\DataTypeTok{pars =} \KeywordTok{list}\NormalTok{(}\DataTypeTok{outpch=}\OtherTok{NA}\NormalTok{)) }
\KeywordTok{stripchart}\NormalTok{(}\DataTypeTok{x =}\NormalTok{ cereal2}\OperatorTok{$}\NormalTok{sodium }\OperatorTok{~}\NormalTok{cereal2}\OperatorTok{$}\NormalTok{Shelf, }\DataTypeTok{lwd =} \DecValTok{1}\NormalTok{, }\DataTypeTok{col =} \StringTok{"red"}\NormalTok{, }\DataTypeTok{method =} \StringTok{"jitter"}\NormalTok{, }\DataTypeTok{vertical =} \OtherTok{TRUE}\NormalTok{, }\DataTypeTok{pch =} \DecValTok{1}\NormalTok{, }\DataTypeTok{add =} \OtherTok{TRUE}\NormalTok{)}
\end{Highlighting}
\end{Shaded}

\includegraphics{ChandraShekarBikkanur_Assignment2_files/figure-latex/unnamed-chunk-4-1.pdf}

\begin{Shaded}
\begin{Highlighting}[]
\CommentTok{#install.packages("GGally", dependencies = TRUE)}
\KeywordTok{library}\NormalTok{(GGally)}
\end{Highlighting}
\end{Shaded}

\begin{verbatim}
## Registered S3 method overwritten by 'GGally':
##   method from   
##   +.gg   ggplot2
\end{verbatim}

\begin{verbatim}
## 
## Attaching package: 'GGally'
\end{verbatim}

\begin{verbatim}
## The following object is masked from 'package:dplyr':
## 
##     nasa
\end{verbatim}

\begin{Shaded}
\begin{Highlighting}[]
\KeywordTok{ggparcoord}\NormalTok{(cereal2, }\DataTypeTok{columns =} \DecValTok{2}\OperatorTok{:}\DecValTok{4}\NormalTok{, }\DataTypeTok{groupColumn =} \StringTok{'Shelf'}\NormalTok{, }\DataTypeTok{scale =} \StringTok{'globalminmax'}\NormalTok{, }\DataTypeTok{showPoints =} \OtherTok{TRUE}\NormalTok{, }\DataTypeTok{title =} \StringTok{"Parallel Coordinate Plot for the cereal2 Data"}\NormalTok{, }\DataTypeTok{alphaLines =} \FloatTok{0.6}\NormalTok{, }\DataTypeTok{mapping=}\KeywordTok{aes}\NormalTok{(}\DataTypeTok{color=}\KeywordTok{as.factor}\NormalTok{(Shelf)))}\OperatorTok{+}\StringTok{ }\KeywordTok{xlab}\NormalTok{(}\StringTok{"explanatory variable"}\NormalTok{) }\OperatorTok{+}\StringTok{ }\KeywordTok{ylab}\NormalTok{(}\StringTok{"standardized value"}\NormalTok{) }\OperatorTok{+}\KeywordTok{scale_color_discrete}\NormalTok{(}\StringTok{"Shelf"}\NormalTok{)}
\end{Highlighting}
\end{Shaded}

\includegraphics{ChandraShekarBikkanur_Assignment2_files/figure-latex/unnamed-chunk-5-1.pdf}
From above 2 plots, we can see that the \emph{shelf 2} has cereals with
relatively more \emph{sugar}, \emph{fat} and \emph{sodium} contents.

\textbf{1.2 (1 point):} The response has values of \(1, 2, 3,\) and
\(4\). Explain under what setting would it be desirable to take into
account ordinality, and whether you think that this setting occurs here.
Then estimate a suitable multinomial regression model with linear forms
of the sugar, fat, and sodium variables. Perform LRTs to examine the
importance of each explanatory variable. Show that there are no
significant interactions among the explanatory variables (including an
interaction among all three variables).

\textbf{1.3 (1 point):} Kellogg's Apple Jacks
(\url{http://www.applejacks.com}) is a cereal marketed toward children.
For a serving size of \(28\) grams, its sugar content is \(12\) grams,
fat content is \(0.5\) grams, and sodium content is \(130\) milligrams.
Estimate the shelf probabilities for Apple Jacks.

\textbf{1.4 (1 point):} Construct a plot similar to Figure 3.3 where the
estimated probability for a shelf is on the \emph{y-axis} and the sugar
content is on the \emph{x-axis}. Use the mean overall fat and sodium
content as the corresponding variable values in the model. Interpret the
plot with respect to sugar content.

\textbf{1.5 (1 point):} Estimate odds ratios and calculate corresponding
confidence intervals for each explanatory variable. Relate your
interpretations back to the plots constructed for this exercise.


\end{document}
