\documentclass[]{article}
\usepackage{lmodern}
\usepackage{amssymb,amsmath}
\usepackage{ifxetex,ifluatex}
\usepackage{fixltx2e} % provides \textsubscript
\ifnum 0\ifxetex 1\fi\ifluatex 1\fi=0 % if pdftex
  \usepackage[T1]{fontenc}
  \usepackage[utf8]{inputenc}
\else % if luatex or xelatex
  \ifxetex
    \usepackage{mathspec}
  \else
    \usepackage{fontspec}
  \fi
  \defaultfontfeatures{Ligatures=TeX,Scale=MatchLowercase}
\fi
% use upquote if available, for straight quotes in verbatim environments
\IfFileExists{upquote.sty}{\usepackage{upquote}}{}
% use microtype if available
\IfFileExists{microtype.sty}{%
\usepackage{microtype}
\UseMicrotypeSet[protrusion]{basicmath} % disable protrusion for tt fonts
}{}
\usepackage[margin=1in]{geometry}
\usepackage{hyperref}
\hypersetup{unicode=true,
            pdftitle={W271 - Assignment 1},
            pdfauthor={Chandra Shekar Bikkanur},
            pdfborder={0 0 0},
            breaklinks=true}
\urlstyle{same}  % don't use monospace font for urls
\usepackage{color}
\usepackage{fancyvrb}
\newcommand{\VerbBar}{|}
\newcommand{\VERB}{\Verb[commandchars=\\\{\}]}
\DefineVerbatimEnvironment{Highlighting}{Verbatim}{commandchars=\\\{\}}
% Add ',fontsize=\small' for more characters per line
\usepackage{framed}
\definecolor{shadecolor}{RGB}{248,248,248}
\newenvironment{Shaded}{\begin{snugshade}}{\end{snugshade}}
\newcommand{\AlertTok}[1]{\textcolor[rgb]{0.94,0.16,0.16}{#1}}
\newcommand{\AnnotationTok}[1]{\textcolor[rgb]{0.56,0.35,0.01}{\textbf{\textit{#1}}}}
\newcommand{\AttributeTok}[1]{\textcolor[rgb]{0.77,0.63,0.00}{#1}}
\newcommand{\BaseNTok}[1]{\textcolor[rgb]{0.00,0.00,0.81}{#1}}
\newcommand{\BuiltInTok}[1]{#1}
\newcommand{\CharTok}[1]{\textcolor[rgb]{0.31,0.60,0.02}{#1}}
\newcommand{\CommentTok}[1]{\textcolor[rgb]{0.56,0.35,0.01}{\textit{#1}}}
\newcommand{\CommentVarTok}[1]{\textcolor[rgb]{0.56,0.35,0.01}{\textbf{\textit{#1}}}}
\newcommand{\ConstantTok}[1]{\textcolor[rgb]{0.00,0.00,0.00}{#1}}
\newcommand{\ControlFlowTok}[1]{\textcolor[rgb]{0.13,0.29,0.53}{\textbf{#1}}}
\newcommand{\DataTypeTok}[1]{\textcolor[rgb]{0.13,0.29,0.53}{#1}}
\newcommand{\DecValTok}[1]{\textcolor[rgb]{0.00,0.00,0.81}{#1}}
\newcommand{\DocumentationTok}[1]{\textcolor[rgb]{0.56,0.35,0.01}{\textbf{\textit{#1}}}}
\newcommand{\ErrorTok}[1]{\textcolor[rgb]{0.64,0.00,0.00}{\textbf{#1}}}
\newcommand{\ExtensionTok}[1]{#1}
\newcommand{\FloatTok}[1]{\textcolor[rgb]{0.00,0.00,0.81}{#1}}
\newcommand{\FunctionTok}[1]{\textcolor[rgb]{0.00,0.00,0.00}{#1}}
\newcommand{\ImportTok}[1]{#1}
\newcommand{\InformationTok}[1]{\textcolor[rgb]{0.56,0.35,0.01}{\textbf{\textit{#1}}}}
\newcommand{\KeywordTok}[1]{\textcolor[rgb]{0.13,0.29,0.53}{\textbf{#1}}}
\newcommand{\NormalTok}[1]{#1}
\newcommand{\OperatorTok}[1]{\textcolor[rgb]{0.81,0.36,0.00}{\textbf{#1}}}
\newcommand{\OtherTok}[1]{\textcolor[rgb]{0.56,0.35,0.01}{#1}}
\newcommand{\PreprocessorTok}[1]{\textcolor[rgb]{0.56,0.35,0.01}{\textit{#1}}}
\newcommand{\RegionMarkerTok}[1]{#1}
\newcommand{\SpecialCharTok}[1]{\textcolor[rgb]{0.00,0.00,0.00}{#1}}
\newcommand{\SpecialStringTok}[1]{\textcolor[rgb]{0.31,0.60,0.02}{#1}}
\newcommand{\StringTok}[1]{\textcolor[rgb]{0.31,0.60,0.02}{#1}}
\newcommand{\VariableTok}[1]{\textcolor[rgb]{0.00,0.00,0.00}{#1}}
\newcommand{\VerbatimStringTok}[1]{\textcolor[rgb]{0.31,0.60,0.02}{#1}}
\newcommand{\WarningTok}[1]{\textcolor[rgb]{0.56,0.35,0.01}{\textbf{\textit{#1}}}}
\usepackage{graphicx,grffile}
\makeatletter
\def\maxwidth{\ifdim\Gin@nat@width>\linewidth\linewidth\else\Gin@nat@width\fi}
\def\maxheight{\ifdim\Gin@nat@height>\textheight\textheight\else\Gin@nat@height\fi}
\makeatother
% Scale images if necessary, so that they will not overflow the page
% margins by default, and it is still possible to overwrite the defaults
% using explicit options in \includegraphics[width, height, ...]{}
\setkeys{Gin}{width=\maxwidth,height=\maxheight,keepaspectratio}
\IfFileExists{parskip.sty}{%
\usepackage{parskip}
}{% else
\setlength{\parindent}{0pt}
\setlength{\parskip}{6pt plus 2pt minus 1pt}
}
\setlength{\emergencystretch}{3em}  % prevent overfull lines
\providecommand{\tightlist}{%
  \setlength{\itemsep}{0pt}\setlength{\parskip}{0pt}}
\setcounter{secnumdepth}{0}
% Redefines (sub)paragraphs to behave more like sections
\ifx\paragraph\undefined\else
\let\oldparagraph\paragraph
\renewcommand{\paragraph}[1]{\oldparagraph{#1}\mbox{}}
\fi
\ifx\subparagraph\undefined\else
\let\oldsubparagraph\subparagraph
\renewcommand{\subparagraph}[1]{\oldsubparagraph{#1}\mbox{}}
\fi

%%% Use protect on footnotes to avoid problems with footnotes in titles
\let\rmarkdownfootnote\footnote%
\def\footnote{\protect\rmarkdownfootnote}

%%% Change title format to be more compact
\usepackage{titling}

% Create subtitle command for use in maketitle
\providecommand{\subtitle}[1]{
  \posttitle{
    \begin{center}\large#1\end{center}
    }
}

\setlength{\droptitle}{-2em}

  \title{W271 - Assignment 1}
    \pretitle{\vspace{\droptitle}\centering\huge}
  \posttitle{\par}
    \author{Chandra Shekar Bikkanur}
    \preauthor{\centering\large\emph}
  \postauthor{\par}
    \date{}
    \predate{}\postdate{}
  

\begin{document}
\maketitle

\begin{Shaded}
\begin{Highlighting}[]
\CommentTok{#Loading some libraries for this assignment}
\KeywordTok{library}\NormalTok{(car)}
\KeywordTok{library}\NormalTok{(dplyr)}
\KeywordTok{library}\NormalTok{(Hmisc)}
\KeywordTok{library}\NormalTok{(ggplot2)}
\KeywordTok{library}\NormalTok{(mcprofile)}
\KeywordTok{library}\NormalTok{(nnet)}
\KeywordTok{library}\NormalTok{(MASS)}
\end{Highlighting}
\end{Shaded}

\textbf{Question 1.1:} Use the code from the week 1 live session file
and: (1) redo the exercise for \texttt{n=50,\ n=100,\ n=500}, (2) plot
the graphs, and (3) describe what you have observed from the results.
Use the same \texttt{pi.seq} as in the live session code.

\begin{itemize}
\tightlist
\item
  \textbf{1.1.1:}
\end{itemize}

\begin{Shaded}
\begin{Highlighting}[]
\CommentTok{# function to get the wald true coverage }
\NormalTok{wald.CI.true.coverage =}\StringTok{ }\ControlFlowTok{function}\NormalTok{(pi, }\DataTypeTok{alpha=}\FloatTok{0.05}\NormalTok{, n) \{  }
\NormalTok{  w =}\StringTok{ }\DecValTok{0}\OperatorTok{:}\NormalTok{n}
\NormalTok{  pi.hat =}\StringTok{ }\NormalTok{w}\OperatorTok{/}\NormalTok{n}
\NormalTok{  pmf =}\StringTok{ }\KeywordTok{dbinom}\NormalTok{(}\DataTypeTok{x=}\NormalTok{w, }\DataTypeTok{size=}\NormalTok{n, }\DataTypeTok{prob=}\NormalTok{pi)  }
\NormalTok{  var.wald =}\StringTok{ }\NormalTok{pi.hat}\OperatorTok{*}\NormalTok{(}\DecValTok{1}\OperatorTok{-}\NormalTok{pi.hat)}\OperatorTok{/}\NormalTok{n}
\NormalTok{  wald.CI_lower.bound =}\StringTok{ }\NormalTok{pi.hat }\OperatorTok{-}\StringTok{ }\KeywordTok{qnorm}\NormalTok{(}\DataTypeTok{p =} \DecValTok{1}\OperatorTok{-}\NormalTok{alpha}\OperatorTok{/}\DecValTok{2}\NormalTok{)}\OperatorTok{*}\KeywordTok{sqrt}\NormalTok{(var.wald)}
\NormalTok{  wald.CI_upper.bound =}\StringTok{ }\NormalTok{pi.hat }\OperatorTok{+}\StringTok{ }\KeywordTok{qnorm}\NormalTok{(}\DataTypeTok{p =} \DecValTok{1}\OperatorTok{-}\NormalTok{alpha}\OperatorTok{/}\DecValTok{2}\NormalTok{)}\OperatorTok{*}\KeywordTok{sqrt}\NormalTok{(var.wald)  }
\NormalTok{  covered.pi =}\StringTok{ }\KeywordTok{ifelse}\NormalTok{(}\DataTypeTok{test =}\NormalTok{ pi}\OperatorTok{>}\NormalTok{wald.CI_lower.bound, }\DataTypeTok{yes =} \KeywordTok{ifelse}\NormalTok{(}\DataTypeTok{test =}\NormalTok{ pi}\OperatorTok{<}\NormalTok{wald.CI_upper.bound, }\DataTypeTok{yes=}\DecValTok{1}\NormalTok{, }\DataTypeTok{no=}\DecValTok{0}\NormalTok{), }\DataTypeTok{no=}\DecValTok{0}\NormalTok{)  }
\NormalTok{  wald.CI.true.coverage =}\StringTok{ }\KeywordTok{sum}\NormalTok{(covered.pi}\OperatorTok{*}\NormalTok{pmf)  }
\NormalTok{  wald.df =}\StringTok{ }\KeywordTok{data.frame}\NormalTok{(w, pi.hat, }\KeywordTok{round}\NormalTok{(}\KeywordTok{data.frame}\NormalTok{(pmf, wald.CI_lower.bound,wald.CI_upper.bound),}\DecValTok{4}\NormalTok{), covered.pi)  }
  \KeywordTok{return}\NormalTok{(wald.df)}
\NormalTok{\}}

\CommentTok{# function to get the wald true coverage matrix for different pi values}
\NormalTok{get.wald.CI.true.matrix =}\StringTok{ }\ControlFlowTok{function}\NormalTok{(pi, }\DataTypeTok{alpha=}\FloatTok{0.05}\NormalTok{, n) \{}
\NormalTok{   wald.df =}\StringTok{ }\KeywordTok{wald.CI.true.coverage}\NormalTok{(}\DataTypeTok{pi=}\NormalTok{pi, }\DataTypeTok{alpha=}\FloatTok{0.05}\NormalTok{, }\DataTypeTok{n=}\NormalTok{n)}
\NormalTok{   wald.CI.true.coverage.level =}\StringTok{ }\KeywordTok{sum}\NormalTok{(wald.df}\OperatorTok{$}\NormalTok{covered.pi}\OperatorTok{*}\NormalTok{wald.df}\OperatorTok{$}\NormalTok{pmf)}
\NormalTok{   pi.seq =}\StringTok{ }\KeywordTok{seq}\NormalTok{(}\FloatTok{0.01}\NormalTok{,}\FloatTok{0.99}\NormalTok{, }\DataTypeTok{by=}\FloatTok{0.01}\NormalTok{)}
\NormalTok{   wald.CI.true.matrix =}\StringTok{ }\KeywordTok{matrix}\NormalTok{(}\DataTypeTok{data=}\OtherTok{NA}\NormalTok{,}\DataTypeTok{nrow=}\KeywordTok{length}\NormalTok{(pi.seq),}\DataTypeTok{ncol=}\DecValTok{2}\NormalTok{)}
\NormalTok{   counter=}\DecValTok{1}
   \ControlFlowTok{for}\NormalTok{ (pi }\ControlFlowTok{in}\NormalTok{ pi.seq) \{}
\NormalTok{        wald.df2 =}\StringTok{ }\KeywordTok{wald.CI.true.coverage}\NormalTok{(}\DataTypeTok{pi=}\NormalTok{pi, }\DataTypeTok{alpha=}\FloatTok{0.05}\NormalTok{, }\DataTypeTok{n=}\NormalTok{n)}
\NormalTok{        wald.CI.true.matrix[counter,] =}\StringTok{ }\KeywordTok{c}\NormalTok{(pi,}\KeywordTok{sum}\NormalTok{(wald.df2}\OperatorTok{$}\NormalTok{covered.pi}\OperatorTok{*}\NormalTok{wald.df2}\OperatorTok{$}\NormalTok{pmf))}
\NormalTok{        counter =}\StringTok{ }\NormalTok{counter}\OperatorTok{+}\DecValTok{1}
\NormalTok{       \}}
   \KeywordTok{return}\NormalTok{(wald.CI.true.matrix)}
\NormalTok{\}}
\NormalTok{wald.CI.true.matrix_}\DecValTok{10}\NormalTok{   =}\StringTok{ }\KeywordTok{get.wald.CI.true.matrix}\NormalTok{(}\DataTypeTok{pi =} \FloatTok{0.6}\NormalTok{, }\DataTypeTok{alpha=}\FloatTok{0.05}\NormalTok{, }\DataTypeTok{n=}\DecValTok{10}\NormalTok{)}
\NormalTok{wald.CI.true.matrix_}\DecValTok{50}\NormalTok{   =}\StringTok{ }\KeywordTok{get.wald.CI.true.matrix}\NormalTok{(}\DataTypeTok{pi =} \FloatTok{0.6}\NormalTok{, }\DataTypeTok{alpha=}\FloatTok{0.05}\NormalTok{, }\DataTypeTok{n=}\DecValTok{50}\NormalTok{)}
\NormalTok{wald.CI.true.matrix_}\DecValTok{100}\NormalTok{  =}\StringTok{ }\KeywordTok{get.wald.CI.true.matrix}\NormalTok{(}\DataTypeTok{pi =} \FloatTok{0.6}\NormalTok{, }\DataTypeTok{alpha=}\FloatTok{0.05}\NormalTok{, }\DataTypeTok{n=}\DecValTok{100}\NormalTok{)}
\NormalTok{wald.CI.true.matrix_}\DecValTok{500}\NormalTok{  =}\StringTok{ }\KeywordTok{get.wald.CI.true.matrix}\NormalTok{(}\DataTypeTok{pi =} \FloatTok{0.6}\NormalTok{, }\DataTypeTok{alpha=}\FloatTok{0.05}\NormalTok{, }\DataTypeTok{n=}\DecValTok{500}\NormalTok{)}
\end{Highlighting}
\end{Shaded}

\vspace{0.25in}

\begin{itemize}
\tightlist
\item
  \textbf{1.1.2:}
\end{itemize}

\begin{Shaded}
\begin{Highlighting}[]
\CommentTok{# Plot the true coverage level (for given n and alpha)}
\NormalTok{alpha=}\FloatTok{0.05}
\KeywordTok{par}\NormalTok{(}\DataTypeTok{mfrow=}\KeywordTok{c}\NormalTok{(}\DecValTok{4}\NormalTok{,}\DecValTok{1}\NormalTok{))}
\CommentTok{# n=10}
\KeywordTok{plot}\NormalTok{(}\DataTypeTok{x=}\NormalTok{wald.CI.true.matrix_}\DecValTok{10}\NormalTok{[,}\DecValTok{1}\NormalTok{],}
     \DataTypeTok{y=}\NormalTok{wald.CI.true.matrix_}\DecValTok{10}\NormalTok{[,}\DecValTok{2}\NormalTok{],}
     \DataTypeTok{ylim=}\KeywordTok{c}\NormalTok{(}\DecValTok{0}\NormalTok{,}\DecValTok{1}\NormalTok{),}
     \DataTypeTok{main =} \StringTok{"Wald C.I. True Confidence Level Coverage for n = 10"}\NormalTok{, }\DataTypeTok{xlab=}\KeywordTok{expression}\NormalTok{(pi),}
     \DataTypeTok{ylab=}\StringTok{"True Confidence Level"}\NormalTok{,}
     \DataTypeTok{type=}\StringTok{"l"}\NormalTok{)}
\KeywordTok{abline}\NormalTok{(}\DataTypeTok{h=}\DecValTok{1}\OperatorTok{-}\NormalTok{alpha, }\DataTypeTok{lty=}\StringTok{"dotted"}\NormalTok{)}
\CommentTok{# n=50}
\KeywordTok{plot}\NormalTok{(}\DataTypeTok{x=}\NormalTok{wald.CI.true.matrix_}\DecValTok{50}\NormalTok{[,}\DecValTok{1}\NormalTok{],}
     \DataTypeTok{y=}\NormalTok{wald.CI.true.matrix_}\DecValTok{50}\NormalTok{[,}\DecValTok{2}\NormalTok{],}
     \DataTypeTok{ylim=}\KeywordTok{c}\NormalTok{(}\DecValTok{0}\NormalTok{,}\DecValTok{1}\NormalTok{),}
     \DataTypeTok{main =} \StringTok{"Wald C.I. True Confidence Level Coverage for n = 50"}\NormalTok{, }\DataTypeTok{xlab=}\KeywordTok{expression}\NormalTok{(pi),}
     \DataTypeTok{ylab=}\StringTok{"True Confidence Level"}\NormalTok{,}
     \DataTypeTok{type=}\StringTok{"l"}\NormalTok{)}
\KeywordTok{abline}\NormalTok{(}\DataTypeTok{h=}\DecValTok{1}\OperatorTok{-}\NormalTok{alpha, }\DataTypeTok{lty=}\StringTok{"dotted"}\NormalTok{)}
\CommentTok{# n=100}
\KeywordTok{plot}\NormalTok{(}\DataTypeTok{x=}\NormalTok{wald.CI.true.matrix_}\DecValTok{100}\NormalTok{[,}\DecValTok{1}\NormalTok{],}
     \DataTypeTok{y=}\NormalTok{wald.CI.true.matrix_}\DecValTok{100}\NormalTok{[,}\DecValTok{2}\NormalTok{],}
     \DataTypeTok{ylim=}\KeywordTok{c}\NormalTok{(}\DecValTok{0}\NormalTok{,}\DecValTok{1}\NormalTok{),}
     \DataTypeTok{main =} \StringTok{"Wald C.I. True Confidence Level Coverage for n = 100"}\NormalTok{, }\DataTypeTok{xlab=}\KeywordTok{expression}\NormalTok{(pi),}
     \DataTypeTok{ylab=}\StringTok{"True Confidence Level"}\NormalTok{,}
     \DataTypeTok{type=}\StringTok{"l"}\NormalTok{)}
\KeywordTok{abline}\NormalTok{(}\DataTypeTok{h=}\DecValTok{1}\OperatorTok{-}\NormalTok{alpha, }\DataTypeTok{lty=}\StringTok{"dotted"}\NormalTok{)}
\CommentTok{# n=500}
\KeywordTok{plot}\NormalTok{(}\DataTypeTok{x=}\NormalTok{wald.CI.true.matrix_}\DecValTok{500}\NormalTok{[,}\DecValTok{1}\NormalTok{],}
     \DataTypeTok{y=}\NormalTok{wald.CI.true.matrix_}\DecValTok{500}\NormalTok{[,}\DecValTok{2}\NormalTok{],}
     \DataTypeTok{ylim=}\KeywordTok{c}\NormalTok{(}\DecValTok{0}\NormalTok{,}\DecValTok{1}\NormalTok{),}
     \DataTypeTok{main =} \StringTok{"Wald C.I. True Confidence Level Coverage for n = 500"}\NormalTok{, }\DataTypeTok{xlab=}\KeywordTok{expression}\NormalTok{(pi),}
     \DataTypeTok{ylab=}\StringTok{"True Confidence Level"}\NormalTok{,}
     \DataTypeTok{type=}\StringTok{"l"}\NormalTok{)}
\KeywordTok{abline}\NormalTok{(}\DataTypeTok{h=}\DecValTok{1}\OperatorTok{-}\NormalTok{alpha, }\DataTypeTok{lty=}\StringTok{"dotted"}\NormalTok{)}
\end{Highlighting}
\end{Shaded}

\includegraphics{ChandraShekarBikkanur_Assignment1_files/figure-latex/unnamed-chunk-3-1.pdf}

\vspace{0.25in}

\begin{itemize}
\tightlist
\item
  \textbf{1.1.3:} From above 4 plots, we can see that the confidence
  level coverage is increasing(less fluctuating) as the number of
  samples increases from 10, 50, 100 to 500.
\end{itemize}

\vspace{0.45in}

\textbf{Question 1.2:} (1) Modify the code for the Wilson Interval. (2)
Do the exercise for \texttt{n=10,\ n=50,\ n=100,\ n=500}. (3) Plot the
graphs. (4) Describe what you have observed from the results and compare
the Wald and Wilson intervals based on your results. Use the same
\texttt{pi.seq} as in the live session code.

\vspace{0.25in}

Wilson Confidence Interval for an estimated parameter \(\tilde{\pi}\) is
as follows

\[ 
\tilde{\pi} \pm \frac{Z_{1-\frac{\alpha}{2}} n^{1/2}}{n + Z^2_{1-\frac{\alpha}{2}}} \sqrt{\hat{\pi}(1-\hat{\pi}) + \frac{Z^2_{1-\frac{\alpha}{2}}}{4n}}
\]

where \(\tilde{\pi}\) is defined as

\[
\tilde{\pi} = \frac{w + \frac{1}{2}Z^2_{1-\frac{\alpha}{2}}}{n + Z^2_{1-\frac{\alpha}{2}}}
\]

\begin{itemize}
\tightlist
\item
  \textbf{1.2.1:}
\end{itemize}

\begin{Shaded}
\begin{Highlighting}[]
\CommentTok{# function to get the wilson true coverage}
\NormalTok{wilson.CI.true.coverage =}\StringTok{ }\ControlFlowTok{function}\NormalTok{(pi, }\DataTypeTok{alpha=}\FloatTok{0.05}\NormalTok{, n) \{  }
\NormalTok{  w =}\StringTok{ }\DecValTok{0}\OperatorTok{:}\NormalTok{n}
\NormalTok{  Z =}\StringTok{ }\KeywordTok{qnorm}\NormalTok{(}\DataTypeTok{p =} \DecValTok{1}\OperatorTok{-}\NormalTok{alpha}\OperatorTok{/}\DecValTok{2}\NormalTok{)}
\NormalTok{  Z2 =}\StringTok{ }\NormalTok{Z}\OperatorTok{^}\DecValTok{2}
\NormalTok{  pi.hat =}\StringTok{ }\NormalTok{(w}\OperatorTok{+}\NormalTok{(Z2}\OperatorTok{/}\DecValTok{2}\NormalTok{))}\OperatorTok{/}\NormalTok{(n}\OperatorTok{+}\NormalTok{Z2)}
\NormalTok{  pmf =}\StringTok{ }\KeywordTok{dbinom}\NormalTok{(}\DataTypeTok{x=}\NormalTok{w, }\DataTypeTok{size=}\NormalTok{n, }\DataTypeTok{prob=}\NormalTok{pi)}
\NormalTok{  delta =}\StringTok{ }\NormalTok{(Z}\OperatorTok{*}\KeywordTok{sqrt}\NormalTok{(n))}\OperatorTok{*}\KeywordTok{sqrt}\NormalTok{((pi.hat }\OperatorTok{*}\StringTok{ }\NormalTok{(}\DecValTok{1} \OperatorTok{-}\StringTok{ }\NormalTok{pi.hat)) }\OperatorTok{+}\StringTok{ }\NormalTok{(Z2)}\OperatorTok{/}\NormalTok{(}\DecValTok{4}\OperatorTok{*}\NormalTok{n))}\OperatorTok{/}\NormalTok{(n}\OperatorTok{+}\NormalTok{Z2)}
\NormalTok{  wilson.CI_lower.bound =}\StringTok{ }\NormalTok{pi.hat }\OperatorTok{-}\StringTok{ }\NormalTok{delta}
\NormalTok{  wilson.CI_upper.bound =}\StringTok{ }\NormalTok{pi.hat }\OperatorTok{+}\StringTok{ }\NormalTok{delta  }
\NormalTok{  covered.pi =}\StringTok{ }\KeywordTok{ifelse}\NormalTok{(}\DataTypeTok{test =}\NormalTok{ pi}\OperatorTok{>}\NormalTok{wilson.CI_lower.bound, }\DataTypeTok{yes =} \KeywordTok{ifelse}\NormalTok{(}\DataTypeTok{test =}\NormalTok{ pi}\OperatorTok{<}\NormalTok{wilson.CI_upper.bound, }\DataTypeTok{yes=}\DecValTok{1}\NormalTok{, }\DataTypeTok{no=}\DecValTok{0}\NormalTok{), }\DataTypeTok{no=}\DecValTok{0}\NormalTok{)  }
\NormalTok{  wilson.CI.true.coverage =}\StringTok{ }\KeywordTok{sum}\NormalTok{(covered.pi}\OperatorTok{*}\NormalTok{pmf)  }
\NormalTok{  wilson.df =}\StringTok{ }\KeywordTok{data.frame}\NormalTok{(w, pi.hat, }\KeywordTok{round}\NormalTok{(}\KeywordTok{data.frame}\NormalTok{(pmf, wilson.CI_lower.bound,wilson.CI_upper.bound),}\DecValTok{4}\NormalTok{), covered.pi)  }
  \KeywordTok{return}\NormalTok{(wilson.df)}
\NormalTok{\}}
\end{Highlighting}
\end{Shaded}

\begin{itemize}
\tightlist
\item
  \textbf{1.2.2:}
\end{itemize}

\begin{Shaded}
\begin{Highlighting}[]
\CommentTok{# function to get the wilson true coverage matrix for different pi values}
\NormalTok{get.wilson.CI.true.matrix =}\StringTok{ }\ControlFlowTok{function}\NormalTok{(pi, }\DataTypeTok{alpha=}\FloatTok{0.05}\NormalTok{, n) \{}
\NormalTok{   wilson.df =}\StringTok{ }\KeywordTok{wilson.CI.true.coverage}\NormalTok{(}\DataTypeTok{pi=}\NormalTok{pi, }\DataTypeTok{alpha=}\FloatTok{0.05}\NormalTok{, }\DataTypeTok{n=}\NormalTok{n)}
\NormalTok{   wilson.CI.true.coverage.level =}\StringTok{ }\KeywordTok{sum}\NormalTok{(wilson.df}\OperatorTok{$}\NormalTok{covered.pi}\OperatorTok{*}\NormalTok{wilson.df}\OperatorTok{$}\NormalTok{pmf)}
\NormalTok{   pi.seq =}\StringTok{ }\KeywordTok{seq}\NormalTok{(}\FloatTok{0.01}\NormalTok{,}\FloatTok{0.99}\NormalTok{, }\DataTypeTok{by=}\FloatTok{0.01}\NormalTok{)}
\NormalTok{   wilson.CI.true.matrix =}\StringTok{ }\KeywordTok{matrix}\NormalTok{(}\DataTypeTok{data=}\OtherTok{NA}\NormalTok{,}\DataTypeTok{nrow=}\KeywordTok{length}\NormalTok{(pi.seq),}\DataTypeTok{ncol=}\DecValTok{2}\NormalTok{)}
\NormalTok{   counter=}\DecValTok{1}
   \ControlFlowTok{for}\NormalTok{ (pi }\ControlFlowTok{in}\NormalTok{ pi.seq) \{}
\NormalTok{        wilson.df2 =}\StringTok{ }\KeywordTok{wilson.CI.true.coverage}\NormalTok{(}\DataTypeTok{pi=}\NormalTok{pi, }\DataTypeTok{alpha=}\FloatTok{0.05}\NormalTok{, }\DataTypeTok{n=}\NormalTok{n)}
\NormalTok{        wilson.CI.true.matrix[counter,] =}\StringTok{ }\KeywordTok{c}\NormalTok{(pi,}\KeywordTok{sum}\NormalTok{(wilson.df2}\OperatorTok{$}\NormalTok{covered.pi}\OperatorTok{*}\NormalTok{wilson.df2}\OperatorTok{$}\NormalTok{pmf))}
\NormalTok{        counter =}\StringTok{ }\NormalTok{counter}\OperatorTok{+}\DecValTok{1}
\NormalTok{       \}}
   \KeywordTok{return}\NormalTok{(wilson.CI.true.matrix)}
\NormalTok{\}}
\NormalTok{wilson.CI.true.matrix_}\DecValTok{10}\NormalTok{   =}\StringTok{ }\KeywordTok{get.wilson.CI.true.matrix}\NormalTok{(}\DataTypeTok{pi =} \FloatTok{0.6}\NormalTok{, }\DataTypeTok{alpha=}\FloatTok{0.05}\NormalTok{, }\DataTypeTok{n=}\DecValTok{10}\NormalTok{)}
\NormalTok{wilson.CI.true.matrix_}\DecValTok{50}\NormalTok{   =}\StringTok{ }\KeywordTok{get.wilson.CI.true.matrix}\NormalTok{(}\DataTypeTok{pi =} \FloatTok{0.6}\NormalTok{, }\DataTypeTok{alpha=}\FloatTok{0.05}\NormalTok{, }\DataTypeTok{n=}\DecValTok{50}\NormalTok{)}
\NormalTok{wilson.CI.true.matrix_}\DecValTok{100}\NormalTok{   =}\StringTok{ }\KeywordTok{get.wilson.CI.true.matrix}\NormalTok{(}\DataTypeTok{pi =} \FloatTok{0.6}\NormalTok{, }\DataTypeTok{alpha=}\FloatTok{0.05}\NormalTok{, }\DataTypeTok{n=}\DecValTok{100}\NormalTok{)}
\NormalTok{wilson.CI.true.matrix_}\DecValTok{500}\NormalTok{   =}\StringTok{ }\KeywordTok{get.wilson.CI.true.matrix}\NormalTok{(}\DataTypeTok{pi =} \FloatTok{0.6}\NormalTok{, }\DataTypeTok{alpha=}\FloatTok{0.05}\NormalTok{, }\DataTypeTok{n=}\DecValTok{500}\NormalTok{)}
\end{Highlighting}
\end{Shaded}

\begin{itemize}
\tightlist
\item
  \textbf{1.2.3:}
\end{itemize}

\begin{Shaded}
\begin{Highlighting}[]
\CommentTok{# Plot the true coverage level (for given n and alpha)}
\KeywordTok{par}\NormalTok{(}\DataTypeTok{mfrow=}\KeywordTok{c}\NormalTok{(}\DecValTok{4}\NormalTok{,}\DecValTok{1}\NormalTok{))}
\CommentTok{# n=10}
\KeywordTok{plot}\NormalTok{(}\DataTypeTok{x=}\NormalTok{wilson.CI.true.matrix_}\DecValTok{10}\NormalTok{[,}\DecValTok{1}\NormalTok{],}
     \DataTypeTok{y=}\NormalTok{wilson.CI.true.matrix_}\DecValTok{10}\NormalTok{[,}\DecValTok{2}\NormalTok{],}
     \DataTypeTok{ylim=}\KeywordTok{c}\NormalTok{(}\DecValTok{0}\NormalTok{,}\DecValTok{1}\NormalTok{),}
     \DataTypeTok{main =} \StringTok{"Wilson C.I. True Confidence Level Coverage for n = 10"}\NormalTok{, }\DataTypeTok{xlab=}\KeywordTok{expression}\NormalTok{(pi),}
     \DataTypeTok{ylab=}\StringTok{"True Confidence Level"}\NormalTok{,}
     \DataTypeTok{type=}\StringTok{"l"}\NormalTok{)}
\KeywordTok{abline}\NormalTok{(}\DataTypeTok{h=}\DecValTok{1}\OperatorTok{-}\NormalTok{alpha, }\DataTypeTok{lty=}\StringTok{"dotted"}\NormalTok{)}
\CommentTok{# n=50}
\KeywordTok{plot}\NormalTok{(}\DataTypeTok{x=}\NormalTok{wilson.CI.true.matrix_}\DecValTok{50}\NormalTok{[,}\DecValTok{1}\NormalTok{],}
     \DataTypeTok{y=}\NormalTok{wilson.CI.true.matrix_}\DecValTok{50}\NormalTok{[,}\DecValTok{2}\NormalTok{],}
     \DataTypeTok{ylim=}\KeywordTok{c}\NormalTok{(}\DecValTok{0}\NormalTok{,}\DecValTok{1}\NormalTok{),}
     \DataTypeTok{main =} \StringTok{"Wilson C.I. True Confidence Level Coverage for n = 50"}\NormalTok{, }\DataTypeTok{xlab=}\KeywordTok{expression}\NormalTok{(pi),}
     \DataTypeTok{ylab=}\StringTok{"True Confidence Level"}\NormalTok{,}
     \DataTypeTok{type=}\StringTok{"l"}\NormalTok{)}
\KeywordTok{abline}\NormalTok{(}\DataTypeTok{h=}\DecValTok{1}\OperatorTok{-}\NormalTok{alpha, }\DataTypeTok{lty=}\StringTok{"dotted"}\NormalTok{)}
\CommentTok{# n=100}
\KeywordTok{plot}\NormalTok{(}\DataTypeTok{x=}\NormalTok{wilson.CI.true.matrix_}\DecValTok{100}\NormalTok{[,}\DecValTok{1}\NormalTok{],}
     \DataTypeTok{y=}\NormalTok{wilson.CI.true.matrix_}\DecValTok{100}\NormalTok{[,}\DecValTok{2}\NormalTok{],}
     \DataTypeTok{ylim=}\KeywordTok{c}\NormalTok{(}\DecValTok{0}\NormalTok{,}\DecValTok{1}\NormalTok{),}
     \DataTypeTok{main =} \StringTok{"Wilson C.I. True Confidence Level Coverage for n = 100"}\NormalTok{, }\DataTypeTok{xlab=}\KeywordTok{expression}\NormalTok{(pi),}
     \DataTypeTok{ylab=}\StringTok{"True Confidence Level"}\NormalTok{,}
     \DataTypeTok{type=}\StringTok{"l"}\NormalTok{)}
\KeywordTok{abline}\NormalTok{(}\DataTypeTok{h=}\DecValTok{1}\OperatorTok{-}\NormalTok{alpha, }\DataTypeTok{lty=}\StringTok{"dotted"}\NormalTok{)}
\CommentTok{# n=500}
\KeywordTok{plot}\NormalTok{(}\DataTypeTok{x=}\NormalTok{wilson.CI.true.matrix_}\DecValTok{500}\NormalTok{[,}\DecValTok{1}\NormalTok{],}
     \DataTypeTok{y=}\NormalTok{wilson.CI.true.matrix_}\DecValTok{500}\NormalTok{[,}\DecValTok{2}\NormalTok{],}
     \DataTypeTok{ylim=}\KeywordTok{c}\NormalTok{(}\DecValTok{0}\NormalTok{,}\DecValTok{1}\NormalTok{),}
     \DataTypeTok{main =} \StringTok{"Wilson C.I. True Confidence Level Coverage for n = 500"}\NormalTok{, }\DataTypeTok{xlab=}\KeywordTok{expression}\NormalTok{(pi),}
     \DataTypeTok{ylab=}\StringTok{"True Confidence Level"}\NormalTok{,}
     \DataTypeTok{type=}\StringTok{"l"}\NormalTok{)}
\KeywordTok{abline}\NormalTok{(}\DataTypeTok{h=}\DecValTok{1}\OperatorTok{-}\NormalTok{alpha, }\DataTypeTok{lty=}\StringTok{"dotted"}\NormalTok{)}
\end{Highlighting}
\end{Shaded}

\includegraphics{ChandraShekarBikkanur_Assignment1_files/figure-latex/unnamed-chunk-6-1.pdf}

\vspace{0.25in}

\begin{itemize}
\tightlist
\item
  \textbf{1.2.4:} From the plots for wald confidence coverage and wilson
  confidence coverage, we can see that the wilson confidence coverage
  shows the 95\% coverage relatively more accurate for less number of
  samples than wald coverage. \newpage
\end{itemize}

\textbf{Question 2:} Using the \texttt{placekick.BW.csv} data,
\texttt{Distance}, \texttt{Weather}, \texttt{Wind15},
\texttt{Temperature}, \texttt{Grass}, \texttt{Pressure}, and
\texttt{Ice} as explanatory variables in a logistic regression model,
complete the following:

\begin{itemize}
\item
  \begin{enumerate}
  \def\labelenumi{(\alph{enumi})}
  \tightlist
  \item
    Estimate the model and properly define the indicator variables used
    within it.
  \end{enumerate}
\item
  \begin{enumerate}
  \def\labelenumi{(\alph{enumi})}
  \setcounter{enumi}{1}
  \tightlist
  \item
    The authors use ``Sun'' as the base level category for Weather,
    which is not the default level that R uses. Describe how ``Sun'' can
    be specified as the base level in R.
  \end{enumerate}
\item
  \begin{enumerate}
  \def\labelenumi{(\alph{enumi})}
  \setcounter{enumi}{2}
  \tightlist
  \item
    Perform LRTs for all explanatory variables to evaluate their
    importance within the model. Discuss the results.
  \end{enumerate}
\item
  \begin{enumerate}
  \def\labelenumi{(\alph{enumi})}
  \setcounter{enumi}{3}
  \tightlist
  \item
    Estimate an appropriate odds ratio for distance, and compute the
    corresponding confidence interval. Interpret the odds ratio.
  \end{enumerate}
\end{itemize}

\begin{Shaded}
\begin{Highlighting}[]
\CommentTok{# Load the data}
\NormalTok{placekick <-}\StringTok{ }\KeywordTok{read.csv}\NormalTok{(}\StringTok{"placekick.BW.csv"}\NormalTok{, }\DataTypeTok{header =} \OtherTok{TRUE}\NormalTok{, }\DataTypeTok{sep =} \StringTok{","}\NormalTok{)}
\KeywordTok{head}\NormalTok{(placekick, }\DecValTok{10}\NormalTok{)}
\end{Highlighting}
\end{Shaded}

\begin{verbatim}
##      GameNum Kicker Good Distance Weather Wind15 Temperature Grass
## 1  2002-0101 Bryant    Y       29     Sun      0        Nice     1
## 2  2002-0101 Bryant    Y       33     Sun      0        Nice     1
## 3  2002-0101 Cortez    N       25     Sun      0        Nice     1
## 4  2002-0101 Cortez    Y       23     Sun      0        Nice     1
## 5  2002-0101 Cortez    N       48     Sun      0        Nice     1
## 6  2002-0101 Cortez    Y       33     Sun      0        Nice     1
## 7  2002-0101 Cortez    Y       36     Sun      0        Nice     1
## 8  2002-0102   Hall    Y       34     Sun      0         Hot     0
## 9  2002-0102   Hall    Y       45     Sun      0         Hot     0
## 10 2002-0102   Hall    N       48     Sun      0         Hot     0
##    Pressure Ice
## 1         N   0
## 2         N   0
## 3         N   0
## 4         N   0
## 5         N   0
## 6         N   0
## 7         Y   0
## 8         N   0
## 9         N   0
## 10        N   0
\end{verbatim}

\begin{Shaded}
\begin{Highlighting}[]
\CommentTok{#check the structure of the data frame}
\KeywordTok{str}\NormalTok{(placekick)}
\end{Highlighting}
\end{Shaded}

\begin{verbatim}
## 'data.frame':    2003 obs. of  10 variables:
##  $ GameNum    : Factor w/ 523 levels "2002-0101","2002-0102",..: 1 1 1 1 1 1 1 2 2 2 ...
##  $ Kicker     : Factor w/ 52 levels "Akers","Andersen",..: 8 8 13 13 13 13 13 27 27 27 ...
##  $ Good       : Factor w/ 2 levels "N","Y": 2 2 1 2 1 2 2 2 2 1 ...
##  $ Distance   : int  29 33 25 23 48 33 36 34 45 48 ...
##  $ Weather    : Factor w/ 4 levels "Clouds","Inside",..: 4 4 4 4 4 4 4 4 4 4 ...
##  $ Wind15     : int  0 0 0 0 0 0 0 0 0 0 ...
##  $ Temperature: Factor w/ 3 levels "Cold","Hot","Nice": 3 3 3 3 3 3 3 2 2 2 ...
##  $ Grass      : int  1 1 1 1 1 1 1 0 0 0 ...
##  $ Pressure   : Factor w/ 2 levels "N","Y": 1 1 1 1 1 1 2 1 1 1 ...
##  $ Ice        : int  0 0 0 0 0 0 0 0 0 0 ...
\end{verbatim}

\begin{itemize}
\tightlist
\item
  \textbf{2.a:}
\end{itemize}

\begin{Shaded}
\begin{Highlighting}[]
\NormalTok{mod.fit <-}\StringTok{ }\KeywordTok{glm}\NormalTok{(Good }\OperatorTok{~}\StringTok{ }\NormalTok{Distance }\OperatorTok{+}\StringTok{ }\NormalTok{Weather }\OperatorTok{+}\StringTok{ }\NormalTok{Wind15 }\OperatorTok{+}\StringTok{ }\NormalTok{Temperature }\OperatorTok{+}\StringTok{ }\NormalTok{Grass }\OperatorTok{+}\StringTok{ }\NormalTok{Pressure }\OperatorTok{+}\StringTok{ }\NormalTok{Ice, }\DataTypeTok{family =} \KeywordTok{binomial}\NormalTok{(}\DataTypeTok{link =}\NormalTok{ logit), }\DataTypeTok{data =}\NormalTok{ placekick )}
\KeywordTok{summary}\NormalTok{(mod.fit)}
\end{Highlighting}
\end{Shaded}

\begin{verbatim}
## 
## Call:
## glm(formula = Good ~ Distance + Weather + Wind15 + Temperature + 
##     Grass + Pressure + Ice, family = binomial(link = logit), 
##     data = placekick)
## 
## Deviance Residuals: 
##     Min       1Q   Median       3Q      Max  
## -2.6804   0.2599   0.4360   0.7148   1.8698  
## 
## Coefficients:
##                  Estimate Std. Error z value Pr(>|z|)    
## (Intercept)      5.740185   0.369597  15.531   <2e-16 ***
## Distance        -0.109600   0.007188 -15.249   <2e-16 ***
## WeatherInside   -0.083030   0.214711  -0.387   0.6990    
## WeatherSnowRain -0.444193   0.217852  -2.039   0.0415 *  
## WeatherSun      -0.247582   0.139642  -1.773   0.0762 .  
## Wind15          -0.243777   0.175527  -1.389   0.1649    
## TemperatureHot   0.250013   0.247540   1.010   0.3125    
## TemperatureNice  0.234932   0.181461   1.295   0.1954    
## Grass           -0.328435   0.160050  -2.052   0.0402 *  
## PressureY        0.270174   0.262809   1.028   0.3039    
## Ice             -0.876133   0.451251  -1.942   0.0522 .  
## ---
## Signif. codes:  0 '***' 0.001 '**' 0.01 '*' 0.05 '.' 0.1 ' ' 1
## 
## (Dispersion parameter for binomial family taken to be 1)
## 
##     Null deviance: 2104.0  on 2002  degrees of freedom
## Residual deviance: 1791.3  on 1992  degrees of freedom
## AIC: 1813.3
## 
## Number of Fisher Scoring iterations: 5
\end{verbatim}

From above logisitc model summary, we can see that \texttt{Clouds} is
the base level indicator for \texttt{Weather} variable; \texttt{Cold} is
the base level indicator for \texttt{Temperature} variable and
\texttt{N} is the base level indicator for \texttt{Pressure} variable.
Also, we can see that \textbf{distance}, \textbf{SnowRain}, \textbf{Sun}
(levels of Weather) \textbf{Grass} and \textbf{Ice} are statistically
significant explantory variables for a \textbf{Good} kick.

\begin{itemize}
\tightlist
\item
  \textbf{2.b:} Let us check the exisitng base level for
  \texttt{Weather} variable
\end{itemize}

\begin{Shaded}
\begin{Highlighting}[]
\CommentTok{# Check the existing base level for Weather}
\KeywordTok{levels}\NormalTok{(placekick}\OperatorTok{$}\NormalTok{Weather)}
\end{Highlighting}
\end{Shaded}

\begin{verbatim}
## [1] "Clouds"   "Inside"   "SnowRain" "Sun"
\end{verbatim}

We see that \texttt{Clouds}to be the base level for \texttt{Weather}
factor variable. Now, we will relevel the dataframe with \texttt{Sun} as
the base level.

\begin{Shaded}
\begin{Highlighting}[]
\CommentTok{# For Weather variable, relevel the base level to "Sun"}
\NormalTok{placekick <-}\StringTok{ }\KeywordTok{within}\NormalTok{(placekick, Weather <-}\StringTok{ }\KeywordTok{relevel}\NormalTok{(Weather, }\DataTypeTok{ref =} \StringTok{"Sun"}\NormalTok{))}
\KeywordTok{levels}\NormalTok{(placekick}\OperatorTok{$}\NormalTok{Weather)}
\end{Highlighting}
\end{Shaded}

\begin{verbatim}
## [1] "Sun"      "Clouds"   "Inside"   "SnowRain"
\end{verbatim}

We can see from above updated levels, ``Sun'' became the current base
level for \texttt{Weather} variable. Now let us re run the above
logistic model with this updated base level to see difference in the
indicator variables.

\begin{Shaded}
\begin{Highlighting}[]
\NormalTok{mod.fit2 <-}\StringTok{ }\KeywordTok{glm}\NormalTok{(Good }\OperatorTok{~}\StringTok{ }\NormalTok{Distance }\OperatorTok{+}\StringTok{ }\NormalTok{Weather }\OperatorTok{+}\StringTok{ }\NormalTok{Wind15 }\OperatorTok{+}\StringTok{ }\NormalTok{Temperature }\OperatorTok{+}\StringTok{ }\NormalTok{Grass }\OperatorTok{+}\StringTok{ }\NormalTok{Pressure }\OperatorTok{+}\StringTok{ }\NormalTok{Ice, }\DataTypeTok{family =} \KeywordTok{binomial}\NormalTok{(}\DataTypeTok{link =}\NormalTok{ logit), }\DataTypeTok{data =}\NormalTok{ placekick )}
\KeywordTok{summary}\NormalTok{(mod.fit2)}
\end{Highlighting}
\end{Shaded}

\begin{verbatim}
## 
## Call:
## glm(formula = Good ~ Distance + Weather + Wind15 + Temperature + 
##     Grass + Pressure + Ice, family = binomial(link = logit), 
##     data = placekick)
## 
## Deviance Residuals: 
##     Min       1Q   Median       3Q      Max  
## -2.6804   0.2599   0.4360   0.7148   1.8698  
## 
## Coefficients:
##                  Estimate Std. Error z value Pr(>|z|)    
## (Intercept)      5.492602   0.370141  14.839   <2e-16 ***
## Distance        -0.109600   0.007188 -15.249   <2e-16 ***
## WeatherClouds    0.247582   0.139642   1.773   0.0762 .  
## WeatherInside    0.164553   0.215062   0.765   0.4442    
## WeatherSnowRain -0.196611   0.219015  -0.898   0.3693    
## Wind15          -0.243777   0.175527  -1.389   0.1649    
## TemperatureHot   0.250013   0.247540   1.010   0.3125    
## TemperatureNice  0.234932   0.181461   1.295   0.1954    
## Grass           -0.328435   0.160050  -2.052   0.0402 *  
## PressureY        0.270174   0.262809   1.028   0.3039    
## Ice             -0.876133   0.451251  -1.942   0.0522 .  
## ---
## Signif. codes:  0 '***' 0.001 '**' 0.01 '*' 0.05 '.' 0.1 ' ' 1
## 
## (Dispersion parameter for binomial family taken to be 1)
## 
##     Null deviance: 2104.0  on 2002  degrees of freedom
## Residual deviance: 1791.3  on 1992  degrees of freedom
## AIC: 1813.3
## 
## Number of Fisher Scoring iterations: 5
\end{verbatim}

\begin{itemize}
\tightlist
\item
  \textbf{2.c:}
\end{itemize}

To test the significance of explantory variables, let us run the
Likelihood Ratio Test using \texttt{Anova} function from \texttt{car}
package

\begin{Shaded}
\begin{Highlighting}[]
\KeywordTok{Anova}\NormalTok{(mod.fit, }\DataTypeTok{test=}\StringTok{"LR"}\NormalTok{)}
\end{Highlighting}
\end{Shaded}

\begin{verbatim}
## Analysis of Deviance Table (Type II tests)
## 
## Response: Good
##             LR Chisq Df Pr(>Chisq)    
## Distance     294.341  1    < 2e-16 ***
## Weather        5.670  3    0.12884    
## Wind15         1.898  1    0.16833    
## Temperature    1.723  2    0.42254    
## Grass          4.314  1    0.03781 *  
## Pressure       1.088  1    0.29682    
## Ice            3.698  1    0.05448 .  
## ---
## Signif. codes:  0 '***' 0.001 '**' 0.01 '*' 0.05 '.' 0.1 ' ' 1
\end{verbatim}

From above \texttt{LRT} test results, we can see that \texttt{Distance},
\texttt{Grass} and \texttt{Ice} are significant explanatory variables
that could explain the logisitc model.

\begin{itemize}
\tightlist
\item
  \textbf{2.d:} Estimated Odds Ratio for a change of c units in x is: \[
  \widehat {OR} = \frac{Odds_{x+c}}{Odds_x} = exp(c\hat \beta_1)
  \]
  \((1-\alpha) 100 \% \space Wald \space C.I \space for \space estimated \space OR :\)
  \[
   exp(c\beta_1 \pm c Z_{1-\frac{\alpha}{2}} \sqrt{\widehat{Var}(\hat \beta_1)})
  \]
\end{itemize}

\begin{Shaded}
\begin{Highlighting}[]
\CommentTok{# Estmated Odds Ratio for 1 unit distance change}
\NormalTok{OR_distance <-}\StringTok{ }\KeywordTok{exp}\NormalTok{(mod.fit}\OperatorTok{$}\NormalTok{coefficients[}\DecValTok{2}\NormalTok{]) }
\NormalTok{OR_distance}
\end{Highlighting}
\end{Shaded}

\begin{verbatim}
##  Distance 
## 0.8961929
\end{verbatim}

Let us evaluate the Wald Confidence Interval for OR.

\begin{Shaded}
\begin{Highlighting}[]
\NormalTok{b.ci <-}\StringTok{ }\KeywordTok{confint.default}\NormalTok{(mod.fit, }\StringTok{"Distance"}\NormalTok{, }\FloatTok{0.95}\NormalTok{)}
\KeywordTok{cat}\NormalTok{(}\StringTok{'95% C.I. for OR:}\CharTok{\textbackslash{}n}\StringTok{'}\NormalTok{, b.ci)}
\end{Highlighting}
\end{Shaded}

\begin{verbatim}
## 95% C.I. for OR:
##  -0.1236869 -0.09551228
\end{verbatim}

\begin{Shaded}
\begin{Highlighting}[]
\KeywordTok{cat}\NormalTok{(}\StringTok{'For c = -15 yards, 95% C.I. for OR:}\CharTok{\textbackslash{}n}\StringTok{'}\NormalTok{, }\KeywordTok{rev}\NormalTok{(}\KeywordTok{exp}\NormalTok{(}\OperatorTok{-}\DecValTok{15} \OperatorTok{*}\StringTok{ }\NormalTok{b.ci)))}
\end{Highlighting}
\end{Shaded}

\begin{verbatim}
## For c = -15 yards, 95% C.I. for OR:
##  4.189931 6.393642
\end{verbatim}

We see from above results, every decreae in 15 yards in distance will
result in an increase in the odds of success by 4 to 6 times.

\newpage

\textbf{Question 3.1:} Examine the data and conduct EDA

\begin{Shaded}
\begin{Highlighting}[]
\CommentTok{# Load the admissions data}
\NormalTok{admissions <-}\StringTok{ }\KeywordTok{read.csv}\NormalTok{(}\StringTok{"admissions.csv"}\NormalTok{, }\DataTypeTok{header =} \OtherTok{TRUE}\NormalTok{, }\DataTypeTok{sep =} \StringTok{","}\NormalTok{)}
\KeywordTok{head}\NormalTok{(admissions)}
\end{Highlighting}
\end{Shaded}

\begin{verbatim}
##   X admit gre  gpa rank
## 1 1     0 380 3.61    3
## 2 2     1 660 3.67    3
## 3 3     1 800 4.00    1
## 4 4     1 640 3.19    4
## 5 5     0 520 2.93    4
## 6 6     1 760 3.00    2
\end{verbatim}

\begin{Shaded}
\begin{Highlighting}[]
\CommentTok{#Check the summary of admissions dataframe}
\KeywordTok{summary}\NormalTok{(admissions)}
\end{Highlighting}
\end{Shaded}

\begin{verbatim}
##        X             admit             gre             gpa       
##  Min.   :  1.0   Min.   :0.0000   Min.   :220.0   Min.   :2.260  
##  1st Qu.:100.8   1st Qu.:0.0000   1st Qu.:520.0   1st Qu.:3.130  
##  Median :200.5   Median :0.0000   Median :580.0   Median :3.395  
##  Mean   :200.5   Mean   :0.3175   Mean   :587.7   Mean   :3.390  
##  3rd Qu.:300.2   3rd Qu.:1.0000   3rd Qu.:660.0   3rd Qu.:3.670  
##  Max.   :400.0   Max.   :1.0000   Max.   :800.0   Max.   :4.000  
##       rank      
##  Min.   :1.000  
##  1st Qu.:2.000  
##  Median :2.000  
##  Mean   :2.485  
##  3rd Qu.:3.000  
##  Max.   :4.000
\end{verbatim}

\begin{Shaded}
\begin{Highlighting}[]
\CommentTok{# Check the structure of admissions dataframe}
\KeywordTok{str}\NormalTok{(admissions)}
\end{Highlighting}
\end{Shaded}

\begin{verbatim}
## 'data.frame':    400 obs. of  5 variables:
##  $ X    : int  1 2 3 4 5 6 7 8 9 10 ...
##  $ admit: int  0 1 1 1 0 1 1 0 1 0 ...
##  $ gre  : int  380 660 800 640 520 760 560 400 540 700 ...
##  $ gpa  : num  3.61 3.67 4 3.19 2.93 3 2.98 3.08 3.39 3.92 ...
##  $ rank : int  3 3 1 4 4 2 1 2 3 2 ...
\end{verbatim}

\begin{Shaded}
\begin{Highlighting}[]
\CommentTok{# Update the rank variable data type to factor from int}
\NormalTok{admissions}\OperatorTok{$}\NormalTok{rank <-}\StringTok{ }\KeywordTok{factor}\NormalTok{(admissions}\OperatorTok{$}\NormalTok{rank)}
\KeywordTok{levels}\NormalTok{(admissions}\OperatorTok{$}\NormalTok{rank)}
\end{Highlighting}
\end{Shaded}

\begin{verbatim}
## [1] "1" "2" "3" "4"
\end{verbatim}

\begin{Shaded}
\begin{Highlighting}[]
\KeywordTok{xtabs}\NormalTok{(}\OperatorTok{~}\NormalTok{admit }\OperatorTok{+}\StringTok{ }\NormalTok{rank, }\DataTypeTok{data =}\NormalTok{ admissions)}
\end{Highlighting}
\end{Shaded}

\begin{verbatim}
##      rank
## admit  1  2  3  4
##     0 28 97 93 55
##     1 33 54 28 12
\end{verbatim}

\begin{Shaded}
\begin{Highlighting}[]
\KeywordTok{cat}\NormalTok{(}\StringTok{'}\CharTok{\textbackslash{}n}\StringTok{ proportional cross table for admit by rank:}\CharTok{\textbackslash{}n\textbackslash{}n}\StringTok{'}\NormalTok{)}
\end{Highlighting}
\end{Shaded}

\begin{verbatim}
## 
##  proportional cross table for admit by rank:
\end{verbatim}

\begin{Shaded}
\begin{Highlighting}[]
\KeywordTok{round}\NormalTok{(}\KeywordTok{prop.table}\NormalTok{(}\KeywordTok{xtabs}\NormalTok{(}\OperatorTok{~}\NormalTok{admit }\OperatorTok{+}\StringTok{ }\NormalTok{rank, }\DataTypeTok{data =}\NormalTok{ admissions)), }\DecValTok{2}\NormalTok{)}
\end{Highlighting}
\end{Shaded}

\begin{verbatim}
##      rank
## admit    1    2    3    4
##     0 0.07 0.24 0.23 0.14
##     1 0.08 0.14 0.07 0.03
\end{verbatim}

\begin{Shaded}
\begin{Highlighting}[]
\KeywordTok{scatterplotMatrix}\NormalTok{(}\OperatorTok{~}\NormalTok{gre}\OperatorTok{+}\NormalTok{gpa}\OperatorTok{+}\NormalTok{rank}\OperatorTok{|}\NormalTok{admit, }\DataTypeTok{data=}\NormalTok{admissions, }\DataTypeTok{legend =} \OtherTok{TRUE}\NormalTok{, }\DataTypeTok{regLine =} \OtherTok{TRUE}\NormalTok{,  }\DataTypeTok{main=}\StringTok{"admit by explanatory variables"}\NormalTok{)}
\end{Highlighting}
\end{Shaded}

\includegraphics{ChandraShekarBikkanur_Assignment1_files/figure-latex/unnamed-chunk-20-1.pdf}

\textbf{Question 3.2:} Estimate a binary logistic regression using the
following set of explanatory variables: \(gre\), \(gpa\), \(rank\),
\(gre^2\), \(gpa^2\), and \(gre \times gpa\), where \(gre \times gpa\)
denotes the interaction between \(gre\) and \(gpa\) variables

\begin{Shaded}
\begin{Highlighting}[]
\NormalTok{logit.model <-}\StringTok{ }\KeywordTok{glm}\NormalTok{(admit }\OperatorTok{~}\StringTok{ }\NormalTok{gre }\OperatorTok{+}\StringTok{ }\NormalTok{gpa }\OperatorTok{+}\StringTok{ }\NormalTok{rank }\OperatorTok{+}\StringTok{ }\KeywordTok{I}\NormalTok{(gre}\OperatorTok{^}\DecValTok{2}\NormalTok{) }\OperatorTok{+}\StringTok{ }\KeywordTok{I}\NormalTok{(gpa}\OperatorTok{^}\DecValTok{2}\NormalTok{) }\OperatorTok{+}\StringTok{ }\NormalTok{gre}\OperatorTok{:}\NormalTok{gpa, }\DataTypeTok{data =}\NormalTok{ admissions, }\DataTypeTok{family =} \KeywordTok{binomial}\NormalTok{(}\DataTypeTok{link=}\StringTok{"logit"}\NormalTok{))}
\KeywordTok{summary}\NormalTok{(logit.model)}
\end{Highlighting}
\end{Shaded}

\begin{verbatim}
## 
## Call:
## glm(formula = admit ~ gre + gpa + rank + I(gre^2) + I(gpa^2) + 
##     gre:gpa, family = binomial(link = "logit"), data = admissions)
## 
## Deviance Residuals: 
##     Min       1Q   Median       3Q      Max  
## -1.5502  -0.8754  -0.6297   1.1187   2.1888  
## 
## Coefficients:
##               Estimate Std. Error z value Pr(>|z|)    
## (Intercept) -7.325e+00  9.065e+00  -0.808 0.419012    
## gre          1.860e-02  1.184e-02   1.571 0.116136    
## gpa         -1.777e-01  4.952e+00  -0.036 0.971371    
## rank2       -7.130e-01  3.202e-01  -2.227 0.025958 *  
## rank3       -1.341e+00  3.474e-01  -3.861 0.000113 ***
## rank4       -1.595e+00  4.221e-01  -3.780 0.000157 ***
## I(gre^2)     3.070e-06  8.216e-06   0.374 0.708624    
## I(gpa^2)     6.699e-01  7.625e-01   0.878 0.379690    
## gre:gpa     -5.888e-03  3.196e-03  -1.842 0.065475 .  
## ---
## Signif. codes:  0 '***' 0.001 '**' 0.01 '*' 0.05 '.' 0.1 ' ' 1
## 
## (Dispersion parameter for binomial family taken to be 1)
## 
##     Null deviance: 499.98  on 399  degrees of freedom
## Residual deviance: 454.90  on 391  degrees of freedom
## AIC: 472.9
## 
## Number of Fisher Scoring iterations: 4
\end{verbatim}

\textbf{Question 3.3:} Test the hypothesis that GRE has no effect on
admission using the likelihood ratio test

Let us quickly check the above model's explanatory variables'
significance using \texttt{LRT}

\begin{Shaded}
\begin{Highlighting}[]
\KeywordTok{Anova}\NormalTok{(logit.model, }\DataTypeTok{test=}\StringTok{"LR"}\NormalTok{)}
\end{Highlighting}
\end{Shaded}

\begin{verbatim}
## Analysis of Deviance Table (Type II tests)
## 
## Response: admit
##          LR Chisq Df Pr(>Chisq)    
## gre        0.3687  1    0.54373    
## gpa        0.0238  1    0.87749    
## rank      21.8244  3  7.095e-05 ***
## I(gre^2)   0.1383  1    0.70994    
## I(gpa^2)   0.7620  1    0.38269    
## gre:gpa    3.4119  1    0.06473 .  
## ---
## Signif. codes:  0 '***' 0.001 '**' 0.01 '*' 0.05 '.' 0.1 ' ' 1
\end{verbatim}

From above \texttt{LRT} results, \texttt{gre} seems to not significant.
However, rank and gre-gpa interaction look significant Let us build
another model (H0) to test the null hypothesis that \texttt{gre} has no
effect on admission.

\begin{Shaded}
\begin{Highlighting}[]
\NormalTok{logit.model2 <-}\StringTok{ }\KeywordTok{glm}\NormalTok{(admit }\OperatorTok{~}\StringTok{   }\NormalTok{gpa }\OperatorTok{+}\StringTok{ }\NormalTok{rank }\OperatorTok{+}\StringTok{ }\KeywordTok{I}\NormalTok{(gre}\OperatorTok{^}\DecValTok{2}\NormalTok{) }\OperatorTok{+}\StringTok{ }\KeywordTok{I}\NormalTok{(gpa}\OperatorTok{^}\DecValTok{2}\NormalTok{) }\OperatorTok{+}\StringTok{ }\NormalTok{gre}\OperatorTok{:}\NormalTok{gpa, }\DataTypeTok{data =}\NormalTok{ admissions, }\DataTypeTok{family =} \KeywordTok{binomial}\NormalTok{(}\DataTypeTok{link=}\StringTok{"logit"}\NormalTok{))}
\end{Highlighting}
\end{Shaded}

Now, let us compare these two models aginst \texttt{LRT} test.

\begin{Shaded}
\begin{Highlighting}[]
\KeywordTok{anova}\NormalTok{(logit.model2, logit.model, }\StringTok{"LRT"}\NormalTok{)}
\end{Highlighting}
\end{Shaded}

\begin{verbatim}
## Analysis of Deviance Table
## 
## Model 1: admit ~ gpa + rank + I(gre^2) + I(gpa^2) + gre:gpa
## Model 2: admit ~ gre + gpa + rank + I(gre^2) + I(gpa^2) + gre:gpa
##   Resid. Df Resid. Dev Df Deviance
## 1       392     457.54            
## 2       391     454.90  1   2.6419
\end{verbatim}

From above results, we can say that \texttt{gre} has no significance in
admission. However, gre-gpa interaction has some effect on admission.

\textbf{Question 3.4:} What is the estimated effect of college GPA on
admission?

\textbf{Question 3.5:} Construct the confidence interval for the
admission probability for the students with \(GPA = 3.3\),
\(GRE = 720\), and \(rank=1\)

\begin{Shaded}
\begin{Highlighting}[]
\NormalTok{K <-}\StringTok{ }\KeywordTok{matrix}\NormalTok{ (}\DataTypeTok{data=} \KeywordTok{c}\NormalTok{(}\DecValTok{1}\NormalTok{, }\DecValTok{720}\NormalTok{, }\FloatTok{3.3}\NormalTok{, }\DecValTok{0}\NormalTok{,}\DecValTok{0}\NormalTok{,}\DecValTok{0}\NormalTok{, }\DecValTok{720}\OperatorTok{*}\DecValTok{720}\NormalTok{, }\FloatTok{3.3}\OperatorTok{*}\FloatTok{3.3}\NormalTok{, }\DecValTok{720}\OperatorTok{*}\FloatTok{3.3}\NormalTok{), }\DataTypeTok{nrow =}\DecValTok{1}\NormalTok{, }\DataTypeTok{ncol =} \DecValTok{9}\NormalTok{)}
\NormalTok{linear.model.fit <-}\StringTok{ }\KeywordTok{mcprofile}\NormalTok{(logit.model, }\DataTypeTok{CM =}\NormalTok{ K)}
\NormalTok{ci.logit.profile <-}\StringTok{ }\KeywordTok{confint}\NormalTok{(linear.model.fit, }\DataTypeTok{level =} \FloatTok{0.95}\NormalTok{)}

\KeywordTok{round}\NormalTok{(}\KeywordTok{exp}\NormalTok{(ci.logit.profile}\OperatorTok{$}\NormalTok{confint)}\OperatorTok{/}\NormalTok{(}\DecValTok{1} \OperatorTok{+}\StringTok{ }\KeywordTok{exp}\NormalTok{(ci.logit.profile}\OperatorTok{$}\NormalTok{confint)), }\DecValTok{2}\NormalTok{)}
\end{Highlighting}
\end{Shaded}

\begin{verbatim}
##   lower upper
## 1  0.44  0.74
\end{verbatim}

From above confidence interval, we can see that the probability of admit
is 0.44 \textless{} \(\hat\pi\) \textless{} .74

\newpage

\textbf{Question 4.1:} Estimate a binary logistic regression with
\texttt{lfp}, which is a binary variable recoding the participation of
the females in the sample, as the dependent variable. The set of
explanatory variables includes \texttt{age}, \texttt{inc}, \texttt{wc},
\texttt{hc}, \texttt{lwg}, \texttt{totalKids}, and a quadratic term of
\texttt{age}, called \texttt{age\_squared}, where \texttt{totalKids} is
the total number of children up to age \(18\) and is equal to the sum of
\texttt{k5} and \texttt{k618}.

\begin{Shaded}
\begin{Highlighting}[]
\KeywordTok{data}\NormalTok{(Mroz) }\CommentTok{# Load th data}
\NormalTok{Mroz}\OperatorTok{$}\NormalTok{totalKids <-}\StringTok{ }\NormalTok{Mroz}\OperatorTok{$}\NormalTok{k5 }\OperatorTok{+}\StringTok{ }\NormalTok{Mroz}\OperatorTok{$}\NormalTok{k618 }\CommentTok{# Add another column totalKids to Mroz }
\KeywordTok{str}\NormalTok{(Mroz)}
\end{Highlighting}
\end{Shaded}

\begin{verbatim}
## 'data.frame':    753 obs. of  9 variables:
##  $ lfp      : Factor w/ 2 levels "no","yes": 2 2 2 2 2 2 2 2 2 2 ...
##  $ k5       : int  1 0 1 0 1 0 0 0 0 0 ...
##  $ k618     : int  0 2 3 3 2 0 2 0 2 2 ...
##  $ age      : int  32 30 35 34 31 54 37 54 48 39 ...
##  $ wc       : Factor w/ 2 levels "no","yes": 1 1 1 1 2 1 2 1 1 1 ...
##  $ hc       : Factor w/ 2 levels "no","yes": 1 1 1 1 1 1 1 1 1 1 ...
##  $ lwg      : num  1.2102 0.3285 1.5141 0.0921 1.5243 ...
##  $ inc      : num  10.9 19.5 12 6.8 20.1 ...
##  $ totalKids: int  1 2 4 3 3 0 2 0 2 2 ...
\end{verbatim}

Let us create a logisitc model for \texttt{lfp}

\begin{Shaded}
\begin{Highlighting}[]
\NormalTok{mroz.glm <-}\StringTok{ }\KeywordTok{glm}\NormalTok{(lfp }\OperatorTok{~}\StringTok{ }\NormalTok{age }\OperatorTok{+}\StringTok{ }\NormalTok{wc }\OperatorTok{+}\StringTok{ }\NormalTok{hc }\OperatorTok{+}\StringTok{ }\NormalTok{lwg }\OperatorTok{+}\StringTok{ }\NormalTok{inc }\OperatorTok{+}\StringTok{ }\NormalTok{totalKids }\OperatorTok{+}\StringTok{ }\KeywordTok{I}\NormalTok{(age}\OperatorTok{^}\DecValTok{2}\NormalTok{), }\DataTypeTok{family =} \KeywordTok{binomial}\NormalTok{(}\DataTypeTok{link =}\NormalTok{ logit), }\DataTypeTok{data =}\NormalTok{ Mroz)}
\KeywordTok{summary}\NormalTok{(mroz.glm) }\CommentTok{# Summary of the logit model}
\end{Highlighting}
\end{Shaded}

\begin{verbatim}
## 
## Call:
## glm(formula = lfp ~ age + wc + hc + lwg + inc + totalKids + I(age^2), 
##     family = binomial(link = logit), data = Mroz)
## 
## Deviance Residuals: 
##     Min       1Q   Median       3Q      Max  
## -1.8342  -1.1669   0.6773   1.0079   2.0614  
## 
## Coefficients:
##              Estimate Std. Error z value Pr(>|z|)    
## (Intercept) -5.294073   2.281551  -2.320 0.020320 *  
## age          0.318014   0.109463   2.905 0.003670 ** 
## wcyes        0.666013   0.218074   3.054 0.002258 ** 
## hcyes        0.098260   0.198970   0.494 0.621417    
## lwg          0.549976   0.145506   3.780 0.000157 ***
## inc         -0.034561   0.007922  -4.363 1.28e-05 ***
## totalKids   -0.222490   0.063849  -3.485 0.000493 ***
## I(age^2)    -0.004114   0.001272  -3.233 0.001224 ** 
## ---
## Signif. codes:  0 '***' 0.001 '**' 0.01 '*' 0.05 '.' 0.1 ' ' 1
## 
## (Dispersion parameter for binomial family taken to be 1)
## 
##     Null deviance: 1029.75  on 752  degrees of freedom
## Residual deviance:  952.02  on 745  degrees of freedom
## AIC: 968.02
## 
## Number of Fisher Scoring iterations: 4
\end{verbatim}

From above model's summary e can see that all explanatory variables
except husband education are significant for the model estimation.

\textbf{Question 4.2:} Is the age effect statistically significant?

\begin{Shaded}
\begin{Highlighting}[]
\KeywordTok{Anova}\NormalTok{(mroz.glm, }\DataTypeTok{test=}\StringTok{"LR"}\NormalTok{)}
\end{Highlighting}
\end{Shaded}

\begin{verbatim}
## Analysis of Deviance Table (Type II tests)
## 
## Response: lfp
##           LR Chisq Df Pr(>Chisq)    
## age         8.6144  1  0.0033351 ** 
## wc          9.5398  1  0.0020107 ** 
## hc          0.2439  1  0.6213914    
## lwg        15.0213  1  0.0001063 ***
## inc        21.0740  1  4.419e-06 ***
## totalKids  12.4267  1  0.0004232 ***
## I(age^2)   10.7487  1  0.0010435 ** 
## ---
## Signif. codes:  0 '***' 0.001 '**' 0.01 '*' 0.05 '.' 0.1 ' ' 1
\end{verbatim}

Yes. \texttt{age} of the women has statistical significance for the
model.

\textbf{Question 4.3:} What is the effect of a decrease in age by \(5\)
years on the odds of labor force participation for a female who was
\(45\) years of age.

Odds Ratio for \(logit(\pi) = \beta_0 + \beta_1x_1 + \beta_2x_1^2:\)

\[
exp(c\beta_1+c\beta_2(2x_1+c))
\]

\begin{Shaded}
\begin{Highlighting}[]
\NormalTok{c <-}\StringTok{ }\DecValTok{-5}
\NormalTok{x1 <-}\StringTok{ }\DecValTok{45}
\NormalTok{beta1 <-}\StringTok{ }\NormalTok{mroz.glm}\OperatorTok{$}\NormalTok{coefficients[}\DecValTok{2}\NormalTok{]}
\NormalTok{beta2 <-}\StringTok{ }\NormalTok{mroz.glm}\OperatorTok{$}\NormalTok{coefficients[}\DecValTok{8}\NormalTok{]}
\NormalTok{OR <-}\StringTok{ }\KeywordTok{exp}\NormalTok{(c}\OperatorTok{*}\NormalTok{beta1 }\OperatorTok{+}\StringTok{ }\NormalTok{(c}\OperatorTok{*}\NormalTok{beta2)}\OperatorTok{*}\NormalTok{(}\DecValTok{2}\OperatorTok{*}\NormalTok{x1 }\OperatorTok{+}\StringTok{ }\NormalTok{c))}
\NormalTok{OR}
\end{Highlighting}
\end{Shaded}

\begin{verbatim}
##      age 
## 1.171602
\end{verbatim}

From above result, we can say that the Odds Ratio of \texttt{lfp} is
1.17 times more likely when the age is 45 and decreases by 5 years.

\textbf{Question 4.4:} Estimate the profile likelihood confidence
interval of the probability of labor force participation for females who
were \(40\) years old, had income equal to \(20\), did not attend
college, had log wage equal to 1, and did not have children.

\begin{Shaded}
\begin{Highlighting}[]
\NormalTok{alpha =}\StringTok{ }\FloatTok{0.5}
\NormalTok{predict.data <-}\StringTok{ }\KeywordTok{data.frame}\NormalTok{(}\DataTypeTok{age =} \DecValTok{40}\NormalTok{,}
                                       \DataTypeTok{inc =} \DecValTok{20}\NormalTok{,}
                           \DataTypeTok{wc =} \KeywordTok{factor}\NormalTok{(}\StringTok{"no"}\NormalTok{), }
                           \DataTypeTok{hc =} \KeywordTok{factor}\NormalTok{(}\StringTok{"no"}\NormalTok{),}
                           \DataTypeTok{lwg =} \DecValTok{1}\NormalTok{,}
                                       \DataTypeTok{totalKids =} \DecValTok{0}\NormalTok{)}

\NormalTok{linear.pred <-}\StringTok{ }\KeywordTok{predict}\NormalTok{(}\DataTypeTok{object =}\NormalTok{ mroz.glm, }\DataTypeTok{newdata =}\NormalTok{ predict.data,}
                      \DataTypeTok{type =} \StringTok{"link"}\NormalTok{, }\DataTypeTok{se =} \OtherTok{TRUE}\NormalTok{)}
\NormalTok{pi.hat <-}\StringTok{ }\KeywordTok{exp}\NormalTok{(linear.pred}\OperatorTok{$}\NormalTok{fit)}\OperatorTok{/}\NormalTok{(}\DecValTok{1}\OperatorTok{+}\KeywordTok{exp}\NormalTok{(linear.pred}\OperatorTok{$}\NormalTok{fit))}
\NormalTok{CI.lin.pred <-}\StringTok{ }\NormalTok{linear.pred}\OperatorTok{$}\NormalTok{fit }\OperatorTok{+}\StringTok{ }\KeywordTok{qnorm}\NormalTok{(}\DataTypeTok{p =} \KeywordTok{c}\NormalTok{(alpha}\OperatorTok{/}\DecValTok{2}\NormalTok{,}\DecValTok{1}\OperatorTok{-}\NormalTok{alpha}\OperatorTok{/}\DecValTok{2}\NormalTok{))}\OperatorTok{*}\NormalTok{linear.pred}\OperatorTok{$}\NormalTok{se}
\NormalTok{CI.pi =}\StringTok{ }\KeywordTok{exp}\NormalTok{(CI.lin.pred)}\OperatorTok{/}\NormalTok{(}\DecValTok{1}\OperatorTok{+}\KeywordTok{exp}\NormalTok{(CI.lin.pred))}


\KeywordTok{round}\NormalTok{(}\KeywordTok{data.frame}\NormalTok{(pi.hat, }\DataTypeTok{lower=}\NormalTok{CI.pi[}\DecValTok{1}\NormalTok{], }\DataTypeTok{upper=}\NormalTok{CI.pi[}\DecValTok{2}\NormalTok{]),}\DecValTok{4}\NormalTok{)}
\end{Highlighting}
\end{Shaded}

\begin{verbatim}
##   pi.hat  lower  upper
## 1 0.6688 0.6412 0.6953
\end{verbatim}

\newpage

\hypertarget{maximum-likelihood-2-points}{%
\section{5: Maximum Likelihood (2
points)}\label{maximum-likelihood-2-points}}

\textbf{Question 18 a and b of Chapter 3 (page 192,193)}

For the wheat kernel data (\emph{wheat.csv}), consider a model to
estimate the kernel condition using the density explanatory variable as
a linear term.

\textbf{Question 5.1} Write an R function that computes the
log-likelihood function for the multinomial regression model. Evaluate
the function at the parameter estimates produced by multinom(), and
verify that your computed value is the same as that produced by logLik()
(use the object saved from multinom() within this function).

\textbf{Question 5.2} Maximize the log-likelihood function using optim()
to obtain the MLEs and the estimated covariance matrix. Compare your
answers to what is obtained by multinom(). Note that to obtain starting
values for optim(), one approach is to estimate separate logistic
regression models for \(log \left( \frac{\pi_2}{\pi_1} \right)\) and
\(log \left( \frac{\pi_3}{\pi_1} \right)\). These models are estimated
only for those observations that have the corresponding responses (e.g.,
a \(Y = 1\) or \(Y = 2\) for
\(log \left( \frac{\pi_2}{\pi_1} \right)\)).

\begin{Shaded}
\begin{Highlighting}[]
\NormalTok{wheat <-}\StringTok{ }\KeywordTok{read.csv}\NormalTok{(}\StringTok{"wheat.csv"}\NormalTok{, }\DataTypeTok{header =} \OtherTok{TRUE}\NormalTok{, }\DataTypeTok{sep =} \StringTok{","}\NormalTok{)}
\KeywordTok{head}\NormalTok{(wheat)}
\end{Highlighting}
\end{Shaded}

\begin{verbatim}
##   class  density hardness    size  weight moisture    type
## 1   hrw 1.349253 60.32952 2.30274 24.6480 12.01538 Healthy
## 2   hrw 1.287440 56.08972 2.72573 33.2985 12.17396 Healthy
## 3   hrw 1.233985 43.98743 2.51246 31.7580 11.87949 Healthy
## 4   hrw 1.336534 53.81704 2.27164 32.7060 12.11407 Healthy
## 5   hrw 1.259040 44.39327 2.35478 26.0700 12.06487 Healthy
## 6   hrw 1.300258 48.12066 2.49132 33.2985 12.18577 Healthy
\end{verbatim}

\begin{Shaded}
\begin{Highlighting}[]
\KeywordTok{str}\NormalTok{(wheat)}
\end{Highlighting}
\end{Shaded}

\begin{verbatim}
## 'data.frame':    275 obs. of  7 variables:
##  $ class   : Factor w/ 2 levels "hrw","srw": 1 1 1 1 1 1 1 1 1 1 ...
##  $ density : num  1.35 1.29 1.23 1.34 1.26 ...
##  $ hardness: num  60.3 56.1 44 53.8 44.4 ...
##  $ size    : num  2.3 2.73 2.51 2.27 2.35 ...
##  $ weight  : num  24.6 33.3 31.8 32.7 26.1 ...
##  $ moisture: num  12 12.2 11.9 12.1 12.1 ...
##  $ type    : Factor w/ 3 levels "Healthy","Scab",..: 1 1 1 1 1 1 1 1 1 1 ...
\end{verbatim}

\begin{Shaded}
\begin{Highlighting}[]
\KeywordTok{levels}\NormalTok{(wheat}\OperatorTok{$}\NormalTok{type)}
\end{Highlighting}
\end{Shaded}

\begin{verbatim}
## [1] "Healthy" "Scab"    "Sprout"
\end{verbatim}

\begin{Shaded}
\begin{Highlighting}[]
\NormalTok{mod.fit <-}\StringTok{ }\KeywordTok{multinom}\NormalTok{(}\DataTypeTok{formula =}\NormalTok{ type }\OperatorTok{~}\StringTok{ }\NormalTok{density, }\DataTypeTok{data =}\NormalTok{ wheat)}
\end{Highlighting}
\end{Shaded}

\begin{verbatim}
## # weights:  9 (4 variable)
## initial  value 302.118379 
## iter  10 value 229.769334
## iter  20 value 229.712304
## final  value 229.712290 
## converged
\end{verbatim}

\begin{Shaded}
\begin{Highlighting}[]
\KeywordTok{summary}\NormalTok{(mod.fit)}
\end{Highlighting}
\end{Shaded}

\begin{verbatim}
## Call:
## multinom(formula = type ~ density, data = wheat)
## 
## Coefficients:
##        (Intercept)   density
## Scab      29.37827 -24.56215
## Sprout    19.12165 -15.47633
## 
## Std. Errors:
##        (Intercept)  density
## Scab      3.676892 3.017842
## Sprout    3.337092 2.691429
## 
## Residual Deviance: 459.4246 
## AIC: 467.4246
\end{verbatim}

\begin{Shaded}
\begin{Highlighting}[]
\KeywordTok{Anova}\NormalTok{(mod.fit)}
\end{Highlighting}
\end{Shaded}

\begin{verbatim}
## # weights:  6 (2 variable)
## initial  value 302.118379 
## final  value 301.493429 
## converged
\end{verbatim}

\begin{verbatim}
## Analysis of Deviance Table (Type II tests)
## 
## Response: type
##         LR Chisq Df Pr(>Chisq)    
## density   143.56  2  < 2.2e-16 ***
## ---
## Signif. codes:  0 '***' 0.001 '**' 0.01 '*' 0.05 '.' 0.1 ' ' 1
\end{verbatim}

\begin{Shaded}
\begin{Highlighting}[]
\KeywordTok{logLik}\NormalTok{(mod.fit)}
\end{Highlighting}
\end{Shaded}

\begin{verbatim}
## 'log Lik.' -229.7123 (df=4)
\end{verbatim}

\begin{Shaded}
\begin{Highlighting}[]
\NormalTok{logL =}\StringTok{ }\ControlFlowTok{function}\NormalTok{(beta, x, Y)\{}
\NormalTok{  pi1 <-}\StringTok{ }\KeywordTok{exp}\NormalTok{(beta[}\DecValTok{1}\NormalTok{] }\OperatorTok{+}\StringTok{ }\NormalTok{beta[}\DecValTok{2}\NormalTok{]}\OperatorTok{*}\NormalTok{x)}\OperatorTok{/}\NormalTok{(}\DecValTok{1}\OperatorTok{+}\KeywordTok{exp}\NormalTok{(beta[}\DecValTok{1}\NormalTok{]}\OperatorTok{+}\NormalTok{beta[}\DecValTok{2}\NormalTok{]}\OperatorTok{*}\NormalTok{x)}\OperatorTok{+}\KeywordTok{exp}\NormalTok{(beta[}\DecValTok{3}\NormalTok{]}\OperatorTok{+}\NormalTok{beta[}\DecValTok{4}\NormalTok{]}\OperatorTok{*}\NormalTok{x))}
\NormalTok{  pi2 <-}\StringTok{ }\KeywordTok{exp}\NormalTok{(beta[}\DecValTok{3}\NormalTok{] }\OperatorTok{+}\StringTok{ }\NormalTok{beta[}\DecValTok{4}\NormalTok{]}\OperatorTok{*}\NormalTok{x)}\OperatorTok{/}\NormalTok{(}\DecValTok{1}\OperatorTok{+}\KeywordTok{exp}\NormalTok{(beta[}\DecValTok{1}\NormalTok{]}\OperatorTok{+}\NormalTok{beta[}\DecValTok{2}\NormalTok{]}\OperatorTok{*}\NormalTok{x)}\OperatorTok{+}\KeywordTok{exp}\NormalTok{(beta[}\DecValTok{3}\NormalTok{]}\OperatorTok{+}\NormalTok{beta[}\DecValTok{4}\NormalTok{]}\OperatorTok{*}\NormalTok{x))}
\NormalTok{  v <-}\StringTok{ }\NormalTok{( }\KeywordTok{log}\NormalTok{(pi1))}\OperatorTok{+}\NormalTok{( }\KeywordTok{log}\NormalTok{(pi2))}
  \KeywordTok{sum}\NormalTok{(v)}
\NormalTok{\}}
\NormalTok{betaComb <-}\StringTok{ }\KeywordTok{cbind}\NormalTok{(}\KeywordTok{coefficients}\NormalTok{(mod.fit)[}\DecValTok{1}\NormalTok{,}\DecValTok{1}\OperatorTok{:}\DecValTok{2}\NormalTok{], }\KeywordTok{coefficients}\NormalTok{(mod.fit)[}\DecValTok{2}\NormalTok{,}\DecValTok{1}\OperatorTok{:}\DecValTok{2}\NormalTok{])}
\NormalTok{logLvalue <-}\StringTok{ }\KeywordTok{logL}\NormalTok{(}\DataTypeTok{beta=}\NormalTok{betaComb, }\DataTypeTok{x =}\NormalTok{ wheat}\OperatorTok{$}\NormalTok{density, }\DataTypeTok{Y =} \KeywordTok{as.numeric}\NormalTok{(wheat}\OperatorTok{$}\NormalTok{type))}
\NormalTok{logLvalue}
\end{Highlighting}
\end{Shaded}

\begin{verbatim}
## [1] -809.632
\end{verbatim}


\end{document}
